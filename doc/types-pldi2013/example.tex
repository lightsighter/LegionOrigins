
\section{Circuit Example}
\label{sec:example}

Listing~\ref{lst:circuit_ex} shows a circuit simulation written (with
one exception) in the core Legion language (see
Section~\ref{sec:legioncore}).  The input is a graph of circuit
elements and connecting wires.  To perform parallel
simulation the graph is partitioned into pieces, after which the
simulation is run for a number of time steps.  Each time step has
three phases: calculation of each wire's current, distribution of charge to
the nodes connected to each wire, and updating of node voltages.

The entry function is {\tt simulate\_circuit} (line 15), which is
parameterized on regions {\tt rl}, {\tt rw}, and {\tt rn}.
The wires and nodes are contained in regions {\tt rw} and {\tt rn},
respectively.  The {\tt rl} region holds lists of pointers to the nodes and wires.
When {\tt simulate\_circuit} is invoked it must have both read and
write privileges for the regions {\tt rl}, {\tt rw}, and {\tt rn}
(line 17); the type system enforces that
{\tt simulate\_circuit} only accesses these regions and their
subregions at compile-time.

Partitioning of the graph is done using {\em colorings} built in the
{\tt color\_circuit} function (line 18-19).  A coloring maps region
elements to colors (i.e. integers), with each color corresponding
 to a new subregion
to be created.  Note that it is impossible to pick a single static
partitioning scheme that would do well for all graphs.  Dynamic
partitioning using colorings provides the programmer the flexibility
to make partitioning decisions based on input data.

The {\tt color\_circuit} function returns three colorings.  The first
maps each node to the graph piece that owns it.
The third maps each wire to a graph piece that
owns one of its nodes.  The second coloring captures, for each piece $p$,
the nodes in other pieces on the boundary of $p$, often referred to as
{\em ghost nodes}.  
%We describe how ghost node regions are used in
%conjunction with the owned nodes shortly.  
Note that 
ghost nodes may be on the boundary of multiple pieces and therefore colored
multiple ways.  The second coloring is a {\em multicoloring} because
it may map the same element to multiple colors.

After creating the colorings, the application partitions the node and wire
regions into {\em subregions} according to the colorings (lines 21-25).  For
simplicity, our example only partitions the circuit into two pieces.
Line 21 uses coloring {\tt pc.1} to partition {\tt rn} into {\tt rn0}, and {\tt rn1}.
Subregion {\tt rn0} will contain all locations with a color of 1 in {\tt pc.1},
while {\tt rn1} will contain all locations with a color of 2.  This
partitioning uses a coloring and results in {\em disjoint} subpartitions.
Constraints are introduced into the static environment describing both the
disjointness of the subregions (i.e. {\tt rn0 $*$ rn1}) and the inclusion of
each subregion in the original region (e.g. {\tt rn0 $\leq$ rn}).  Line
23 is similar, but uses a multicoloring, so the subregions may be {\em aliased},
and only the inclusion constraints are introduced into the static environment.
Although multicolorings are not part of the Core Legion described in 
Section~\ref{sec:legioncore}, they can be easily implemented using a separate 
coloring for each color in the multicoloring.

The circuit simulation uses an {\em allocate-then-partition} style of
computation, where a large data structure is first allocated and then
partitioned.  Legion also supports a {\em partition-then-allocate} style,
where empty regions are first partitioned and later populated with
data.  Both are useful; e.g., Sequoia \cite{Fatahalian06} and DPJ
\cite{Bocchino09} support static forms of allocate-then-partition and
partition-then-allocate, respectively.  Legion supports both
approaches dynamically.

After partitioning the circuit into pieces, the application creates
instances of {\tt CircuitPiece} (defined on lines 9-12) for each piece
(lines 28-33).  {\tt CircuitPiece} is a {\em region relationship}, a
bounded existential type.  Region relationships allow the programmer
to {\em pack} a group of regions together and remember properties
about them such as disjointness and subregion relationships (lines 29
and 33).  The type system verifies the properties hold statically when
packing; when region relationships are {\em unpacked} the same
properties are reintroduced into the static checking environment
(lines 40-41).  A key feature underlying the soundness of the Legion
type system is that privileges cannot be packed in a region
relationship---privileges belong to functions. When a function unpacks
a region relationship it must already hold privileges for the new
regions it finds in the packed value.  For example, on line 40
function {\tt execute\_time\_steps} has read/write privileges for
region {\tt rn0} because it has read/write privileges for {\tt rn} and
the {\tt CircuitPiece} region relationship constraints ensure $\tt rn0
\leq rn$ (line 12).

The {\tt execute\_time\_steps} function (lines 37-48)  
also illustrates the importance of using different partitions to give
multiple views onto the same logical region.  Both {\tt
  calc\_new\_currents} and {\tt distribute\_charge} 
use the owned and ghost regions of a piece, which are from different partitions. In
{\tt calc\_new\_currents} these regions only need read
privileges, allowing both instances of {\tt calc\_new\_currents} to be
run in parallel.  In {\tt distribute\_charge} the
privilege is for a reduction which can also be done in parallel
because of the atomic and commutative nature of reductions.  Finally,
the {\tt update\_voltage} function modifies only the owned regions, permitting
each piece to be updated in parallel.  No
single partition of the nodes
describes these data sharing patterns.

%This function also demonstrates the benefit of being able to dynamically
%(re-)discover disjointness.  Safely running both instances of 
%{\tt calc\_new\_currents} in parallel depends on knowing that {\tt rw0} and
%{\tt rw1} in {\tt execute\_time\_steps} are disjoint.  Similarly, parallel
%execution of the instances of {\tt update\_voltage} requires knowing that
%{\tt rn0} is disjoint from {\tt rn1}.  In both cases, this knowledge was
%statically available in {\tt simulate\_circuit} and could have been captured
%in a region relationship, but it would have resulted in much more complicated
%code.
%Instead, an inexpensive dynamic disjointness check is performed by the Legion
%runtime, and the safety of parallel execution follows from the soundness of
%the type system.
The {\tt execute\_time\_steps} function also demonstrates the benefit of dynamically
being able to discover disjointness.  To run the two instances of 
{\tt calc\_new\_currents} in parallel requires knowing that {\tt rw0}
and {\tt rw1} are disjoint.  There is a similar requirement for parallel
execution of the two {\tt update\_voltage} calls with {\tt rn0} and {\tt rn1}.  In
both cases, this knowledge was statically available in {\tt simulate\_circuit}
and could have been captured in a region relationship at the cost
of much more complicated code.  Instead, an inexpensive dynamic disjointness
check by the Legion runtime will discover the parallelism.

Listing~\ref{lst:circuit_leaf} shows the leaf functions for each phase.
Each function iterates over the list of wires or
nodes for its piece.  Each function specifies the region privileges it
requires (lines 2,14,23).  
%These privileges
%are statically checked to match the operations performed inside of
%each function (e.g. read and write).  
In the case of {\em reduce} the
privilege must also specify which reduction function is used
(line 14).

In addition to privileges, functions can also specify {\em coherence}
on regions.  Coherence specifies what updates the function may
see from other functions using the same region.  If not otherwise
specified, coherence defaults to {\em exclusive}, meaning the
function must appear to execute in program order.  Line 14 shows
an example of {\tt atomic}, a relaxed coherence mode requiring
that operations to {\tt rn} and {\tt rg} appear atomic relative
to other functions using those regions.  The most relaxed coherence
mode is {\tt simult}; simultaneous coherence allows concurrent access 
to the region by all functions that are using the region
in a simultaneous mode.  The interaction between {\em tasks} (which are functions chosen by the runtime to execute in parallel) using the same
region with different coherence
modes is formalized in Section~\ref{sec:coherence}.

% This is a description of how the listings should be formatted.
% It can go anywhere before the listings.
\lstset{
  captionpos=b,
  language=Haskell,
  basicstyle=\scriptsize,
  numbers=left,
  numberstyle=\tiny,
  columns=fullflexible,
  stepnumber=1,
  escapechar=\#,
  keepspaces=true,
  literate={<}{{$\langle$}}1 {>}{{$\rangle$}}1,
  morekeywords={function,rr,int,float,bool,isnull,partition,as,downregion,upregion,reads,writes,rdwrs,reduces,read,write,reduce,using,unpack,pack,coloring,multicoloring,color,newcolor,atomic,simultaneous},
  deletekeywords={float,head,min,max}
}

\begin{lstlisting}[float={t},label={lst:circuit_leaf},caption={Circuit Leaf Functions}]
function calc_new_currents[rl,rw,rn,rg] ( ptr_list : wire_list<rl,rw,rn,rg>@rl ), 
      reads(rl,rw,rn,rg), writes(rw) : bool =
  if isnull(ptr_list) then true else
  let wire_node : wire_list<rl,rw,rn,rg> = read(ptr_list) in
  let wire : CircuitWire<rn,rg> = read(wire_node.1) in
  let in_node : CircuitNode = read(wire.1) in
  let out_node : CircuitNode = read(wire.2) in
  let current : float = (in_node.1 - out_node.1) /  wire.3 in 
  let new_wire : CircuitWire<rn,rg> = <wire.1,wire.2,wire.3,current> in
  let _ : CircuitWire<rn,rg>@rw = write(wire_node.1, new_wire) in
      calc_new_currents[rl,rw,rn,rg](wire_node.2)

function distribute_charge[rl,rw,rn,rg] ( ptr_list : wire_list<rl,rw,rn,rg>@rl ), 
      reads(rl,rw,rn), reduces(reduce_charge,rn,rg), atomic(rn,rg) : bool =
  if isnull(ptr_list) then true else
  let wire_node : wire_list<rl,rw,rn,rg> = read(ptr_list) in
  let wire : CircuitWire<rn,rg> = read(wire_node.1) in
  let _ : CircuitNode@rn = reduce(reduce_charge, wire.1, wire.4) in
  let _ : CircuitNode@(rn,rg) = reduce(reduce_charge, wire.2, wire.4) in
      distribute_charge[rl,rw,rn,rg](wire_node.2)

function update_voltage[rl,rn] ( ptr_list : node_list<rl,rn>@rl ), 
      reads(rl,rn), writes(rn) : bool = 
  if isnull(ptr_list) then true else
  let node_node : node_list<rl,rn> = read(ptr_list) in
  let node : CircuitNode = read(node_node.1) in
  let voltage : float = (node.3/node.4) in
  let new_node : CircuitNode = <voltage,node.2,node.3,node.4> in
  let _ : CircuitNode@rn = write(node_node.1, new_node) in
      update_voltage[rl,rn](node_node.2)

-- Reduction function for distribute charge
function reduce_charge ( node : CircuitNode, current : float ) : CircuitNode =
    let new_charge : float = node.3 + current in
        < node.1,new_charge,node.3,node.4>
\end{lstlisting}


