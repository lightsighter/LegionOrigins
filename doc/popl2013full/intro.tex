

\section{Introduction}
\label{sec:intro}

In the past decade machine architecture, particularly at the high
performance end of the spectrum, has undergone a revolution.  The
latest supercomputers are now complex hierarchies of many different
kinds of computing technologies: networks of nodes at the top level,
multiple chips per node, multiple cores within a chip, and, 
recently, multiple accelerators (usually GPUs) per node, which 
can themselves be further decomposed.  We present the operational and static
semantics of Legion\cite{Legion12}, a programming model targeted at providing an
appropriate level of abstraction for programming such machines, one
that is both sufficiently high-level to be portable while still
exposing aspects that are crucial to performance. 

In Legion data is organized in a hierarchy of {\em regions}
and subregions while computation is organized in a hierarchy of {\em
tasks} and subtasks operating on regions.  Regions and tasks interact
through a static system of {\em region privileges} specifying 
operations a task may perform on a region argument (read,
write, or reduce) and  {\em region coherence} annotations that
express what other tasks may do concurrently with
the region (exclusive, atomic, or simultaneous).  We prove the
soundness of the privileges and coherence system and use these two
results to prove a third result: if two {\em siblings} (tasks that
have the same immediate parent task) $t_1$ and $t_2$ are 
{\em non-interfering} (have no ordering requirements for correctness), 
then the any descendant of $t_1$ in the task hierarchy is non-interfering
with any descendant of $t_2$.  This theorem is the basis for correct distributed
task scheduling in a Legion implementation, which is crucial for
performance; a centralized scheduler would be a bottleneck
because of the large communication latencies in the target
class of machines.

Legion programs organize data into a forest of {\em logical regions}.  Logical
regions express {\em locality} (data that will be used together, and therefore should be colocated) 
and {\em independence} (disjoint data that may be operated on in parallel).
Researchers have previously explored language designs with hierarchical decomposition of data to express
locality and independence for high performance programming. Two recent examples, Sequoia \cite{Fatahalian06} and 
Deterministic Parallel Java (DPJ) \cite{Bocchino09}, each provide a mechanism to partition the heap recursively
into a tree of collections of data.  The two designs are different in many aspects, but agree that there is
a single tree-shaped partitioning of data that can be fully checked statically (see Section~\ref{sec:related} for more
discussion of related work).

Our own experience with writing high-performance applications in both 
Sequoia and the current industry standard mix of MPI, shared-memory
threads, and CUDA, has taught us that for many practical situations a
single, static partitioning is insufficient.  In practice, the
best way to partition data is often a function of the data
itself---i.e., the partitions need to be dynamically computed.
Furthermore, it is useful to allow multiple, simultaneous partitions of
the same data, providing different views on to that data. Thus, in Legion the 
handling of regions and partitioning of regions is much more dynamic than in 
previous designs.  In particular,
\begin{itemize}
\item  logical regions are first-class values in Legion
and may be dynamically allocated and stored in data structures;

\item logical regions can be {\em partitioned} into {\em subregions}; the programmer can express arbitrary partitions of
regions;

\item  a logical region may be dynamically partitioned in multiple different ways.
\end{itemize}
One consequence of these decisions is that, through partitioning, a
given datum may belong to multiple different regions.  For example, if
a region $R$ is disjointly partitioned three distinct ways 
and each partition assigns every element in $R$ to a
subregion (partitions need not be total in Legion), then every element
in $R$ is included in three different subregions.  This is why we use
the term {\em logical} regions: language-level regions serve to name
sets of data but do not imply anything about physical layout or placement
in the memory hierarchy.  There
is a separate system of {\em physical regions} at runtime that hold
copies of the data in the language-level logical regions.  In fact,
the Legion implementation routinely maintains multiple distinct
physical regions, each with a copy of the data in the corresponding
logical region, for performance reasons \cite{Legion12}.

A further consequence of placing the same data in multiple logical
regions is that it introduces aliasing which makes it difficult 
to statically determine which computations can be run in parallel.  
One approach is to outlaw aliasing which allows for a static analysis
to discover all available parallelism and schedule it entirely at
compile-time, avoiding any runtime overhead.  Languages such as
Sequoia and DPJ take this approach and allow only a single static partition of any region 
to eliminate region aliasing, at some cost in expressiveness.

% Putting this here so that we can get the code in a good place
\lstset{
  captionpos=b,
  language=Haskell,
  basicstyle=\scriptsize,
  numbers=left,
  numberstyle=\tiny,
  columns=fullflexible,
  stepnumber=1,
  escapechar=\#,
  keepspaces=true,
  literate={<}{{$\langle$}}1 {>}{{$\rangle$}}1,
  morekeywords={function,rr,int,float,bool,isnull,partition,as,downregion,upregion,reads,writes,rdwrs,reduces,read,write,reduce,using,unpack,pack,coloring,multicoloring,color,newcolor,atomic,simultaneous},
  deletekeywords={float,head,min,max}
}
\begin{lstlisting}[float={t},label={lst:circuit_ex},caption={Circuit Simulation}]
--                        <voltage,current,charge,capacitance>
type CircuitNode        = <float,float,float,float>
--                     < owned node, owned or ghost node, resistance, current>
type CircuitWire<rn,rg>  = <CircuitNode@rn, CircuitNode@(rn,rg),float,float>

type node_list<rl,rn>       = < CircuitNode@rn, node_list<rl,rn>@rl >
type wire_list<rl,rw,rn,rg>= < CircuitWire<rn,rg>@rw, wire_list<rl,rw,rn,rg>@rl >

type CircuitPiece<rl,rw,rn> = rr[rpw,rpn,rg]
                            < wire_list<rl,rpw,rpn,rg>@rl, node_list<rl,rpn>@rl >         
                            where rpn #$\le$# rn and rg #$\le$# rn and rpw #$\le$# rw and
                                  rn * rw and rl * rn and rl * rw

-- Simulation initialization and invocation
function simulate_circuit[rl,rw,rn] ( all_nodes : node_list<rl,rn>@rl, 
                            all_wires : wire_list<rl,rw,rn,rn>@rl, steps : int ), 
      reads(rn,rw,rl), writes(rn,rw,rl) : bool = 
  let pc : <coloring(rn),multicoloring(rn),coloring(rw)> 
            = color_circuit[rn,rw,rl](all_nodes,all_wires) in
  -- Disjoint partition for the owned nodes of each piece
  partition rn using pc.1 as rn0,rn1 in
  -- Aliased partition for ghost nodes of each piece
  partition rn using pc.2 as rg0,rg1 in
  -- Disjoint partition for the owned wires of each piece
  partition rw using pc.3 as rw0,rw1 in
  let lists0 : <wire_list<rl,rw0,rn0,rg0>@rl,node_list<rl,rn0>@rl> = 
        build_lists[rl,rw,rn,rw0,rn0,rg0](all_nodes,all_wires,pc.1,pc.2,pc.3,0) in
  let piece0 : CircuitPiece<rl,rw,rn> = 
        pack lists0 as CircuitPiece<rl,rw,rn>[rw0,rn0,rg0] in
  let lists1 : <wire_list<rl,rw1,rn1,rg1>@rl,node_list<rl,rn1>@rl> =
        build_lists[rl,rw,rn,rw1,rn1,rg1](all_nodes,all_wires,pc.1,pc.2,pc.3,1) in
  let piece1 : CircuitPiece<rl,rw,rn> = 
        pack lists1 as CircuitPiece<rl,rw,rn>[rw1,rn1,rg1] in
      execute_time_steps[rl,rw,rn](piece0,piece1,steps)

-- Time Step Loop
function execute_time_steps[rl,rw,rn] ( p0 : CircuitPiece<rl,rw,rn>, 
      p1 : CircuitPiece<rl,rw,rn>, steps : int ) , reads(rn,rw,rl), writes(rn,rw) : bool = 
  if steps #$<$# 1 then true else
  unpack p0 as piece0 : CircuitPiece<rl,rw,rn>[rw0,rn0,rg0] in 
  unpack p1 as piece1 : CircuitPiece<rl,rw,rn>[rw1,rn1,rg1] in
  let _ : bool = calc_new_currents[rl,rw0,rn0,rg0](piece0.1) in
  let _ : bool = calc_new_currents[rl,rw1,rn1,rg1](piece1.1) in
  let _ : bool = distribute_charge[rl,rw0,rn0,rg0](piece0.1) in
  let _ : bool = distribute_charge[rl,rw1,rn1,rg1](piece1.1) in
  let _ : bool = update_voltage[rl,rn0](piece0.2) in
  let _ : bool = update_voltage[rl,rn1](piece1.2) in
      execute_time_steps[rl,rw,rn](p0,p1,steps-1)

function color_circuit[rn,rw,rl] ( all_nodes : node_list<rl,rn>@rl, 
                               all_wires : wire_list<rl,rw,rn>@rl ), 
        reads(rn,rw,rl) : <coloring(rn), multicoloring(rn), coloring(rw)> =  
  -- Invoke programmer chosen coloring algorithm (e.g. METIS)
  -- return owned, ghost, wire colorings

-- Helper method
function build_lists[rl,rw,rn,rpw,rpn,rg] ( nodes : node_list<rl,rn>@rl, 
       wires : wire_list<rl,rw,rn>@rl, oc : coloring(rn), gc : multicoloring(rn), 
       wc : coloring(rw), c : int), reads(rn,rw,rl), writes(rl) 
       : < wire_list<rl,rpw,rpn,rg>@rl, node_list<rl,rpn>@rl > = 
  -- Construct lists of node and wire pointers for the given colorings
\end{lstlisting}

% Moving this to the circuit example section since it fits better there
% and some of this is redundant
%Legion's execution model is that by default, a program's semantics is
%its meaning as a sequential program, which we refer to as the {\em
%program order} execution.  While the
%implementation is not our focus here, if two functions use
%disjoint data (because all of the regions they use are disjoint), then
%Legion may execute them simultaneously.  Also like DPJ,
%Legion has a system of {\em privileges} ({\em read}, {\em write}, and
%{\em reduce}) on regions and an orthogonal system of region {\em
%coherence} modes ({\em excl}, {\em atomic}, and {\em simultaneous})
%that increase the number of situations in which functions can be
%executed in parallel.

The important difference between Legion and the more static approaches
is that Legion allows region aliasing, but does so in a structured way that minimizes
runtime overheads.  The key observation is that
static alias analysis is hard, but dynamic analysis is very easy
and relatively cheap
when done at the granularity of regions instead of individual heap
locations.  Thus, Legion falls between fully static systems, such as
Sequoia and DPJ, and fully dynamic approaches such as transactional
memory.  Legion still has a significant static component; in
particular, the required privileges for function calls
and region
pointer dereferences are checked statically.  The dynamic alias checks
for independence are done at the granularity of regions and one check 
on a region
often serves to prove the independence of many subsequent uses of that region.
In contrast, because transactional memory has no mechanism
for grouping heap data into larger collections, it must test the
overlap between two sets of data on each individual memory location, which is
significantly more expensive.

We begin in Section~\ref{sec:example} with an example program that illustrates
a typical style of programming in Legion and motivates the need for multiple, dynamically computed partitions of
a region.  We define Core Legion, a small language suitable for our formal development, in Section~\ref{sec:legioncore}.
The next four sections each state and prove one of our main results and contributions:
\begin{itemize}
\item We prove the soundness of Legion's privileges system (Section~\ref{sec:soundness}).

\item We use the soundness of privileges to prove the soundness of Legion's region coherence modes (Section~\ref{sec:coherence}).

\item We show that if expressions $e_1$ and $e_2$ are {\em non-interfering} (can be executed in parallel), then subexpressions
$e_1'$ of $e_1$ and $e_2'$ of $e_2$ are also non-interfering (Section~\ref{sec:scheduling}).  This result is the basis
for a hierarchical, distributed scheduler in the Legion implementation which is crucial for high performance
on the target class of machines.
%; on the target class of machines, any centralized
%scheduler would be a serious bottleneck.

\item We give experimental evidence for the Legion design.  On three real-world applications, we show that 
dynamic region pointer checks would be expensive, justifying checking this aspect of the type system statically.
We also show that the cost of checking for non-interference of region privileges is low, showing that a much more expressive
and dynamic language is not incompatible with high performance.

\end{itemize}
Finally, Section~\ref{sec:related} discusses the most closely related work and
Section~\ref{sect:conclusion} concludes.


  










