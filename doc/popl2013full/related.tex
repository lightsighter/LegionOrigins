
\section{Related Work}
\label{sec:related}
Legion is most directly related to Sequoia \cite{Fatahalian06}.  Sequoia is array-based, with
a limited repertoire of ways to partition arrays.  Sequoia is a static language with a single unified hierarchy
of tasks and data; Legion is more dynamic with separate task and region hierarchies.

Deterministic Parallel Java (DPJ) is the only other region-based parallel system of which we are
aware\cite{Bocchino09}.  While there are similarities between DPJ's effects and Legion privileges, 
%and we have reused some DPJ notation, 
there are differences stemming from DPJ's static approach.
Regions in Legion are first-class and can be created, partitioned, packed, and unpacked 
dynamically, allowing programmers to compute data organization at runtime; like Sequoia, DPJ
partition schemes must be statically decided.  Legion allows 
programmers to create multiple partitions of the same region to give different 
views onto the same data, which is not possible in DPJ.  
%DPJ only supports the 
%equivalent to Legion's exclusive and atomic coherence modes \cite{Bocchino11} 
%while Legion provides safe execution even in simultaneous environments.  
There is also a difference in emphasis: DPJ requires shared memory, while Legion 
is designed for distributed heterogeneous machines.

Chapel \cite{Chamberlain:Chapel} and X10 \cite{X1005} also provide some Legion-like facilities.
Chapel's domains and X10's places provide the programmer with a 
mechanism for expressing locality, similar to regions in Legion.  However, domains
and places are not used for independence analysis to discover parallelism.
In contrast, Jade uses annotations to describe
data disjointness,  and like Legion leverages the disjointness information
to discover parallelism, but lacks a region system to name and organize unbounded collections of objects \cite{Rinard98}.  

% Unclear to me what subtleties need to be pointed out here and explained
% to properly differentiate our work
Many efforts use static region systems for  memory management (e.g., \cite{Tofte94, Grossman02}).
Our system is more closely affiliated with dynamic region systems used for expressing locality for performance \cite{Gay01}.

%In addition to region languages with static type systems, there have been several
%dynamic region languages.  Cyclone uses both a static type system and dynamic
%region checks to enforce memory safety properties of C programs\cite{Grossman02}.
%Gay and Aiken introduced RC which reference counts regions dynamically and uses
%a static type system to reason about effeciently garbage collecting regions\cite{Gay01}.

There have been many type and effect systems for ownership types
\cite{Boyapati03} including ones that leverage nested regions for describing
relationships \cite{Clarke02,Cameron07}.  However, ownership type and effect systems
are primarily used for reasoning about determinism in object oriented languages and
don't capture the range of disjointness properties that can be specified in Legion.

Reasoning about disjoint heap data is the strong suit of separation logic \cite{Reynolds02}.  
Concurrent separation logic\cite{Brookes04} has been 
used both to parallelize sequential programs\cite{Raza09,Gotsman07} and to provide 
a mechanism for reasoning about independence\cite{Hayman06}.
While we have borrowed some separation logic notation, we ultimately chose to use a 
privileges system as our primary formalism because separation logic does not easily
support reasoning about the interleaving of operations to non-disjoint regions of memory.

% I'm also not sure how much detail to go into here.  DPJ spends a lot of time
% on these papers, but I'm not sure I understand all the details of these papers.
% DPJ also cites Lu06 here, but I'm not sure if we have to
%Type and effect systems have also been used to reason about regions.  FX presented
%the original type and effect system on regions \cite{Lucassen88}, but was restricted 
%to using a finite number of regions and was incapable of describing nested data
%structures.  

% Do we need to enumerate what these disjointness properties are (DPJ does)
%Type and effect systems have also been used in the context of parallelism to discover
%deadlocks and race conditions \cite{Boyapati02,Abadi06,Jacobs08}, but do not present 
%any mechanism for discovering parallelism.

%Still not sure what we want to put here.
% DPJ sites additional separation logic papers but they didn't seem very similar.
% Let me know if you think we should include them as well.


