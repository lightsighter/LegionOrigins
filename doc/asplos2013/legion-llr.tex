\documentclass{sig-alternate}

\bibliographystyle{plainnat}
\usepackage{amsmath}
%\usepackage{amsthm}
\usepackage{listings}
\usepackage{tikz}
\usepackage{subfigure}
\usepackage{multirow}
\usepackage{pgfplots}
\usepackage{booktabs}
\usepackage{dcolumn}
\usepackage{cancel}

\newcommand{\infrule}[2]{\displaystyle\frac{\displaystyle\strut{#1}}{\displaystyle\strut {#2}}}
\newcommand{\deref}{\ast}
\newcommand{\rread}[1]{\mbox{\em Read}(#1)}
\newcommand{\rwrite}[1]{\mbox{\em Write}(#1)}
\newcommand{\lca}[2]{#1 \sqcup #2}
\newcommand{\rleq}{\leq}
\newcommand{\interval}[1]{\mbox{\em interval}(#1)}
\newcommand{\context}[1]{\mbox{\em context}(#1)}
\newtheorem{theorem}{Theorem} 
\newtheorem{lemma}[theorem]{Lemma}

\begin{document}

\title{Asynchronous High-Performance Language Primitives for Distributed Heterogeneous Architectures}
\numberofauthors{3}
\author{}
\maketitle

\begin{abstract}
%The limit is 12 pages, including bibliography and appendices.
%Deadlines:
%\begin{tabular}{ll}
%Abstract deadline & July 16, 2012 3pm PDT \\
%Paper deadline & July 23, 2012 3pm PDT \\
%Author response & around October 15, 2012 \\
%Author notification & November 9, 2012
%\end{tabular}
Modern parallel architectures are composed of both heterogeneous processors
and complex distributed memory hierarchies.  We present a low-level programming
interface that provides simple primitives for targeting this class of
machines.  Our interface includes mechanisms for the creation and movement
of data through the memory hierarchy as well as a novel synchronization primitive 
called a deferred lock.  To hide the large latencies associated with the target
set of machines the interface is made asynchronous via an event system.  All
operations return an event to be triggered when the operation completes.  Events
can be chained together and used to create dependences between operations.
We describe an implementation of our interface capable of running on
a large supercomputer with both CPU and GPU processors and a high speed interconnect.
We employ several micro-benchmarks to show that our implementation approaches
the performance limits of the underlying hardware.
We also demonstrate that the interface is sufficiently powerful to serve as the foundation 
of an advanced runtime system.  We measure the performance of three real-world applications
that use our system on the Keeneland supercomputer.
\end{abstract}

\section{Introduction}
\label{sect:intro}
Introduction goes here.

\section{Outline}

\begin{enumerate}
\item Abstract
\item Introduction
\item LLR Components
  \begin{enumerate}
  \item Processors
  \item Memories
  \item Events
  \item Locks
  \item Physical Regions
  \end{enumerate}
\item Implementations
  \begin{enumerate}
  \item SMP Version
  \item GASNet/CUDA Version
  \end{enumerate}
\item Microbenchmarks
  \begin{enumerate}
  \item Events
    \begin{enumerate}
    \item Latency (inter- and intra-node)
    \item Throughput (separate tracks and single wide track)
    \end{enumerate}
  \item Locks
    \begin{enumerate}
    \item Latency, Throughput vs. Load
    \item Locality - associate data with lock either implicitly or explicitly, show effect of unfairness
    \end{enumerate}
  \item Reductions
    \begin{enumerate}
    \item Histogram Mini-app - vary data, histogram sizes, use variety of reduction techniques
    \end{enumerate}
  \end{enumerate}
\item Applications
  \begin{enumerate}
  \item Circuit
  \item Fluid
  \item AMR
  \end{enumerate}
\item Application Profiling
  \begin{enumerate}
  \item Events
    \begin{enumerate}
    \item Lifetimes (Storage Cost)
    \item Trigger vs. Query Times
    \item Waiters per Event (individual and aggeregated per-node)
    \end{enumerate}
  \item Locks
    \begin{enumerate}
    \item ???
    \end{enumerate}
  \item Reductions
    \begin{enumerate}
    \item ???
    \end{enumerate}
  \item Deferred Execution
    \begin{enumerate}
    \item Comparison vs. Bulk Synchronous Implementation
    \end{enumerate}
  \end{enumerate}
\item Related Work
\item Conclusion
\item Bibliography
\end{enumerate}

\section{TODO}

\newcommand{\tblhdr}[1]{\multicolumn{1}{c}{\bf #1}}

\begin{tabular}{lll}
\tblhdr{Status} & \tblhdr{Owner} & \tblhdr{Task} \\
DONE & Mike & Code event latency microbenchmark \\
& Mike & Code event throughput microbenchmark \\
& ?? & Code lock latency/throughput microbenchmark \\
& Sean & Code lock locality microbenchmark \\
& ?? & Code histogram microbenchmark \\
& Sean & Detailed event logging \\
& Sean & Bulk-Synchronous Mode: Circuit \\
& Sean & Bulk-Synchronous Mode: Fluid \\
& Mike & Bulk-Synchronous Mode: AMR \\
\end{tabular}



\section{Conclusion}
\label{sec:conclusion}

As large supercomputers continue to scale, the latencies associated
with common operations such as communication and synchronization will
become larger and more variable.  To effectively hide these latencies
future programming systems will need to leverage asynchronous operations
to overlap long latency operations with useful work.  We have
presented a low-level programming interface that is fully asynchronous
including a new synchronization primitive called a deferred lock.
Our interface uses events to compose asynchronous operations.  Clients
can use events to explicitly specify dependences between operations,
leaving the runtime to schedule them in a way that maximizes latency 
hiding while still obeying dependences.

We've presented an implementation of interface that is capable of running
on a large heterogeneous cluster containing both CPUs and GPUs.  We
employed micro-benchmarks to show that our implementation is efficient.
We also demonstrated that our interface is capable of supporting Legion,
an advanced high-level runtime system that fully leverages the asynchronous
features of our interface.  We showed that for three real-world applications
the asynchronous Legion implementations were between 22\% to 135\% faster than
implementations which contained blocking operations.




{
\small
\bibliography{bibliography}
}

\end{document}


