\documentclass{sig-alternate}

\bibliographystyle{plainnat}
\usepackage{amsmath}
%\usepackage{amsthm}
\usepackage{listings}
\usepackage{tikz}
\usepackage{subfigure}
\usepackage{multirow}
\usepackage{pgfplots}
\usepackage{booktabs}
\usepackage{dcolumn}
\usepackage{cancel}

\newcommand{\infrule}[2]{\displaystyle\frac{\displaystyle\strut{#1}}{\displaystyle\strut {#2}}}
\newcommand{\deref}{\ast}
\newcommand{\rread}[1]{\mbox{\em Read}(#1)}
\newcommand{\rwrite}[1]{\mbox{\em Write}(#1)}
\newcommand{\lca}[2]{#1 \sqcup #2}
\newcommand{\rleq}{\leq}
\newcommand{\interval}[1]{\mbox{\em interval}(#1)}
\newcommand{\context}[1]{\mbox{\em context}(#1)}
\newtheorem{theorem}{Theorem} 
\newtheorem{lemma}[theorem]{Lemma}

\begin{document}

\title{Asynchronous High-Performance Language Primitives for Distributed Heterogeneous Architectures}
\numberofauthors{3}
\author{}
\maketitle

\begin{abstract}
%The limit is 12 pages, including bibliography and appendices.
%Deadlines:
%\begin{tabular}{ll}
%Abstract deadline & July 16, 2012 3pm PDT \\
%Paper deadline & July 23, 2012 3pm PDT \\
%Author response & around October 15, 2012 \\
%Author notification & November 9, 2012
%\end{tabular}
Modern parallel architectures are composed of both heterogeneous processors
and complex distributed memory hierarchies.  We present a low-level programming
interface that provides simple primitives for targeting this class of
machines.  Our interface includes mechanisms for the creation and movement
of data through the memory hierarchy as well as a novel synchronization primitive 
called a deferred lock.  To hide the large latencies associated with the target
set of machines the interface is made asynchronous via an event system.  All
operations return an event to be triggered when the operation completes.  Events
can be chained together and used to create dependences between operations.
We describe an implementation of our interface capable of running on
a large supercomputer with both CPU and GPU processors and a high speed interconnect.
We employ several micro-benchmarks to show that our implementation approaches
the performance limits of the underlying hardware.
We also demonstrate that the interface is sufficiently powerful to serve as the foundation 
of an advanced runtime system.  We measure the performance of three real-world applications
that use our system on the Keeneland supercomputer.
\end{abstract}

\section{Introduction}
\label{sect:intro}
Introduction goes here.

\section{Proof Outline}

\begin{itemize}

\item A logical region $L$ is a set of abstract memory locations $a_i, a_j, ... \in A$.

\item A coloring function $c : A \rightarrow C$ is a function from abstract memory locations to ``colors''.

\item A partitioning $P_{L,c} : C \rightarrow 2^A$ of a region $L$ with a coloring function $c$ is a function from a color to a subregion of L, satisfying:

\begin{itemize}

\item $\forall L,c,i,a . a \in L \wedge c(a) = i \leftrightarrow a \in P_{L,c}(i)$

\item $\forall L,c,i_1,i_2 . i_1 \neq i_2 \rightarrow P_{L,c}(i_1) * P_{L,c}(i_2)$

\end{itemize}

\item A static effect $E$ is a set of tuples $\langle L, op \rangle$ where $op \in \{ read, write \} \cup \{ reduce_f : f \text{is a reduction function} \}$.

\item Physical memory locations are represented by the set $m_1, m_2, ... \in M$, along with a mapping function $\alpha : M \rightarrow A$ that describes which abstract location a physical memory location corresponds to.

\item A dynamic trace $D = \langle \hat E, \hat O \rangle$ is a directed acyclic graph whose nodes $\hat E$ are actual memory operations $\langle m, op \rangle$ and edges $\hat O$ describe a partial ordering $\hat e_1 \prec \hat e_2$ of those memoory operations.  (Hmm...  Need notation that makes it clear that the same memory operation can be performed on the same memory address multiple times.)

\item Soundness of effects: If $\vdash t : ^ET$ then the mapping function $\alpha$ and dynamic trace $D = \langle \hat E, \hat O \rangle$ that results from evaluating $t$ have the following properties:

\begin{itemize}

\item $\forall m . \langle m, read \rangle \in \hat E \rightarrow \exists L . \alpha(m) \in L \wedge \langle L, read \rangle \in E$

\item $\forall m . \langle m, write \rangle \in \hat E \rightarrow \exists L . \alpha(m) \in L \wedge \langle L, write \rangle \in E$

\item $\forall m, f . \langle m, reduce_f \rangle \in \hat E \rightarrow ( \exists L . \alpha(m) \in L \wedge \langle L, reduce_f \rangle \in E ) \vee ( \exists L_1, L_2 . \alpha(m) \in L_1 \wedge \langle L_1, read \rangle \in E \wedge \alpha(m) \in L_2 \wedge \langle L_2, write \rangle \in E )$

\end{itemize}

\item Two dynamic subtraces $D_1 = \langle \hat E_1, \hat O_1 \rangle$, and $D_2 = \langle \hat E_2, \hat O_2 \rangle$ are ``memory ordered'' (written $D_1 \prec_D D_2$) within a larger trace $D = \langle \hat E_1 \cup \hat E_2 \cup \hat E', \hat O_1 \cup \hat O_2 \cup \hat O' \rangle$ if $D_2$ sees all the results of $D_1$'s memory operations and $D_1$ sees none of the results of $D_2$'s memory operations:

\begin{tabular}{l@{}l@{}l}
$D_1 \prec_D D_2 \leftrightarrow \forall$ & $\hat e_1 = \langle m_1, op_1 \rangle \in \hat E_1,$ \\
& $\hat e_2 = \langle m_2, op_2 \rangle \in \hat E_2 . \big($ & $m_1 \neq m_2 \vee$  \\
&& $( op_1 = read \wedge op_2 = read ) \vee$ \\
&& $( op_1 = reduce_f \wedge op_2 = reduce_f ) \vee$ \\
&& $( \hat e_1 \prec \hat e_2 \in \hat O' ) \big)$
\end{tabular}

\item Note that if $D_1$ and $D_2$ have no memory addresses in common, then you have $D_1 \prec_D D_2$ and $D_2 \prec_D D_1$ for all D.  Maybe $\prec$ is the wrong symbol to use?

\item Tasks are annotated with a coherence requirements $H_{excl}, H_{atom} \subseteq A$.  The default annotation is $H_{excl} = \bigcup_{\langle L, op \rangle \in E} L, H_{atom} = \emptyset$.

\item The runtime enforces an execution order $\prec_E$ between two tasks $S_1$ and $S_2$ as follows:

\begin{itemize}

\item Strict ordering: when the two tasks have exclusive coherence requirements on two regions that can't be proven disjoint (i.e. $\not\vdash H_{excl_1} * H_{excl_2}$), we enforce $S_1 \prec_E S_2$.

\item Serializability: when the two tasks have atomic coherence requirements on two regions that can't be proven disjoint, we enforce $S_1 \prec_E S_2 \vee S_2 \prec_E S_1$.

\end{itemize}

\item Execution order is stronger than memory order: $(S_1 \prec_E S_2) \rightarrow ( \forall \hat e_1 \in \hat E_1, \hat e_2 \in \hat E_2 . \hat e_1 \prec \hat e_2 ) \rightarrow (\forall D. D_1 \prec_D D_2)$.

\item Coherence of sibling tasks:  If sibling tasks $S_1$ and $S_2$ are program ordered (i.e. $S_1 \prec_P S_2$) within their parent task:

\begin{itemize}

\item Overlap in exclusivity requirements guarantees memory ordering: $H_{excl_1} \cap H_{excl_2} \neq \emptyset \rightarrow D_1 \prec_D D_2$.  (If $\vdash (E_1 \cap H_{excl_1}) * (E_2 \cap H_{excl_2})$, soundness of effects guarantees disjointness of memory addresses and therefore memory ordering.  If not, the runtime enforces execution order and therefore memory ordering.)

\item Overlap in atomic requirements guarantees serializability: $H_{atom_1} \cap H_{atom_2} \neq \emptyset \rightarrow D_1 \prec_D D_2 \vee D_2 \prec_D D_1$.  (Parallels proof above.)

\end{itemize}

\end{itemize}



\section{Conclusion}
We have presented Legion, a programming model and
type system for expressing locality and independence
to target heterogeneous, distributed parallel architectures.  
We implemented both a portable high-level and machine 
abstraction low-level runtime to support the Legion 
programming model.  Our implementation of Legion demonstrated
speedups up to 5.9X on a cluster of GPUs.


{
\small
\bibliography{bibliography}
}

\end{document}


