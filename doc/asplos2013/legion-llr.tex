\documentclass{sig-alternate}

\bibliographystyle{plainnat}
\usepackage{amsmath}
%\usepackage{amsthm}
\usepackage{listings}
\usepackage{tikz}
\usepackage{subfigure}
\usepackage{multirow}
\usepackage{pgfplots}
\usepackage{booktabs}
\usepackage{dcolumn}
\usepackage{cancel}

\newcommand{\infrule}[2]{\displaystyle\frac{\displaystyle\strut{#1}}{\displaystyle\strut {#2}}}
\newcommand{\deref}{\ast}
\newcommand{\rread}[1]{\mbox{\em Read}(#1)}
\newcommand{\rwrite}[1]{\mbox{\em Write}(#1)}
\newcommand{\lca}[2]{#1 \sqcup #2}
\newcommand{\rleq}{\leq}
\newcommand{\interval}[1]{\mbox{\em interval}(#1)}
\newcommand{\context}[1]{\mbox{\em context}(#1)}
\newtheorem{theorem}{Theorem} 
\newtheorem{lemma}[theorem]{Lemma}

\begin{document}

\title{Legion: Expressing Locality and Independence with Logical Regions}
\numberofauthors{3}
\author{}
\maketitle

\begin{abstract}
Last year's limit was 12 pages, including bibliography and appendices.
Deadlines:
\begin{tabular}{ll}
Abstract deadline & July 16, 2012 3pm PDT \\
Paper deadline & July 23, 2012 3pm PDT \\
Author response & around October 15, 2012 \\
Author notification & November 9, 2012
\end{tabular}
\end{abstract}

\section{Introduction}
\label{sect:intro}
Introduction goes here.


\section{Conclusion}
We have presented Legion, a programming model and
type system for expressing locality and independence
to target heterogeneous, distributed parallel architectures.  
We implemented both a portable high-level and machine 
abstraction low-level runtime to support the Legion 
programming model.  Our implementation of Legion demonstrated
speedups up to 5.9X on a cluster of GPUs.


{
\small
\bibliography{bibliography}
}

\end{document}


