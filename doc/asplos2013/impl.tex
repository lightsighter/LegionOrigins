
\section{Runtime Implementation}
\label{sec:impl}

There are currently two implementations of our interface:
one that works only on shared memory machines and another that
runs on larges clusters with both CPUs and GPUs.  The shared-memory
implementation uses the POSIX threads library\cite{PTHREADS} (Pthreads)
and is primarily used for debugging.
The implementation for heterogeneous clusters also uses Pthreads as 
well as the CUDA runtime library for GPUs\cite{CUDA} and the GASNet
API for large clusters\cite{GASNET07}.  

The heterogeneous implementation
models a cluster with CPUs and GPUs as having two types of processors 
and four types of memory.  Every CPU core is presented as a different {\tt CPU} processor
and every GPU is a {\tt GPU} processor.  This matches the scheduling granularity available
in the Pthreads and CUDA APIs respectively.  The first type of memory is the 
system memory that is accessible by every CPU core on a given node.  The second is the 
framebuffer memory on each GPU (and accessible only by that GPU).  The third type
of memory is system memory on a node that has been made accessible to the GPU(s) on
that node as well as the CPU cores, commonly referred to as {\em zero-copy memory}.
The final type of memory is the portion of system memory on each node that has been registered with
the GASNet runtime to allow {\em remote memory access} (RMA) by other nodes in the cluster.
We refer to this memory as {\em GASNet memory}.
%% For every node in the system
%% there is a memory corresponding to the DRAM associated with that node.  For
%% every GPU in the system there are two memories corresponding to the framebuffer
%% memory and the zero-copy memory associated with that GPU \footnote{Zero-copy 
%% memory contains pages that are mapped in both framebuffer and node memory with
%% coherence maintained by the PCI-E protocol.}. The last memory is 
%% a global GASNet memory that represents pages that have been registered 
%% with all nodes in the system for supporting remote memory accesses (RMAs).
%% The GASNet memory provides the illusion of global memory.  CPU processors
%% can directly access their node memory, all zero-copy memories for GPUs on
%% their node, and the GASNet memory.  GPU processors can only access their
%% framebuffer memory and their zero-copy memory.  Copies are permitted
%% between all pairs of memory except between any framebuffer memory and
%% the GASNet memory.

In addition to the RMA capabilities, GASNet provides {\em active messages} for 
inter-node communication.  Active messages consist of a command and payload that
are sent by one node to another without any previous coordination.  Upon arrival
at a destination node, a handler routine is invoked to process an active message.
%The handler may issue a response to the sender, but this is optional, and avoided
%as much as possible in our implementation due to the latencies involved in waiting
%for a response.  
One example of a case where we use active messages involves the {\tt spawn} call on a {\tt Processor}
object.  If the processor is not on the same node as the requestor, the information
%about the task (e.g. the task ID and arguments, the actual target {\tt Processor},
%the prerequisite event (if any), and the completion event ID) 
for the task is sent in an
active message to the node containing the target processor.  The recipient node
unpacks the message, and either places the task on the ready queue for the target
processor or marks it as being deferred until the prerequisite event has triggered.
%No response is sent to the original requestor.  The requestor already knows the event
%that corresponds to the completion of this task (since it was provided in the active
%message) - there is no utility in knowing exactly when the active message itself has
%been processed.

%% To support inter-node operations our implementation relies on GASNet active
%% messages for communication.  Active messages are a command and a payload
%% that are sent remotely and cause a handler to be run on the target node.
%% A simple example of our use of active messages can be illustrated by the 
%% {\tt spawn} call on a {\tt Processor} object.  Consider the case where {\tt spawn}
%% is invoked on a processor that is on a remote node.  This operation is converted into
%% an active message that contains all the information about the task launch.
%% When the active message is handled by the remote node, the task and all its
%% metadata are registered with the processor on which the task was launched.
%% While most of the implementation of our heterogeneous runtime is straight
%% forward, the next three sections describe in greater detail the nuances
%% of the implementation of events, locks, and reduction instances.

%% Our implementation uses a hierarchical model, in which data structures are shared by
%% all threads on the same node (taking advantage of the shared address space and low
%% latencies of using Pthreads mutexes and/or lock-free data structures).

%Due to space constraints, we will not cover every part of our implementation in
%detail, instead limiting our discussion to the three most interesting aspects:
%events, locks, and reduction instances.
Due to space constraints we cover only a small part of the implementation in detail.
We limit our discussion to three interesting aspects:
events, locks, and reduction instances.

\subsection{Event Implementation}
\label{subsec:eventimpl}

Events are created on demand and are {\em owned} by the node on which they
were created.  The creation of an event occurs with no inter-node communcation.
The space of event
handles is divided across the nodes at startup time, allowing each node to assign handles
to new events without the risk of conflicting with another node's assignments.  The static division
of event handles also permits any node to determine the owning node of an event
without any communication.
%No broadcast of
%the event creation is required either, as the static division is sufficient to allow any
%node to correctly determine an event's owning node when (and only if) an operation on that
%event is performed on the other node.

%At event creation time, the owning node allocates a data structure to track the state of the event
%(i.e. triggered or not) as well as record a list of the operations that are known to be dependent
%on the event (e.g. initiation of a copy operation, placing a task on a processor's ready queue, waking
%up a waiting task).  However, the owning node only keeps a list of the dependent operations from the same
%node.  Every other node also allocates a corresponding data structure (the first time the event
%is referenced) to remember if the event has already triggered and to record its local dependent operations.
%Arbitrarily many dependent operations from a
%single node are aggregated into a single {\em event subscription} active message that is sent to the
%owning node.  The owning node keeps a bitmask of which other nodes have subscribed to the event, and when the
%event finally does trigger, a single {\em event trigger} active message is sent to each subscribing node,
%which notes that the event has triggered (in case queries come after the trigger) and executes the list
%of local operations that were dependent on that event.  (In the case that the triggering of the event has
%occurred before the reception of an event subscription, a trigger message is sent immediatel to the 
%new subscriber.)
When a new event is created, the owning node allocates a data structure to track the state of
the event ({\em triggered} or {\em untriggered}) as well as to record a list of operations dependent on the
event from the same node called {\em waiters} (e.g. copy operation or task launch).  The first operation contingent on an event
from a remote node will create the same data structure on the remote node.  A single
{\em event subscription} active message is then sent from the remote node to the owning node indicating
that the remote node should be informed when the event triggers.  Arbitrarily many additional operations on
the remote node can then be aggregated as waiters onto the remote data structure without any additional communication.
When the event does trigger, the owner node notifies all local waiters and
sends a single {\em event trigger} active message to each node from which it has seen an event subscription
to notify remote waiters.
In the case where an event triggers while a subscription message is in flight from a remote node, the owner node will
immediately respond with a trigger message.

The actual triggering of an event may occur on any node.  If it occurs on a node other than the owning
node, an {\em event trigger} active message is sent from the triggering node back to the owning node, which
then forwards that message to all the other subscribed nodes.  The triggering node will automatically
notify its waiters and no message is sent from the owning node back to the triggering node.
%with the exception of the triggering node, which was
%able to execute its locally dependent operations (if any) immediately.  
While a remote trigger of an event can result in
the latency of a triggering operation including two active message flight times, it bounds the number of active
messages required per event to $2N-2$ where $N$ is the number of nodes 
%(which can be much smaller than the number of dependent operations) 
interested in the event.  An alternative
would have been to share the list of subscribers so that the triggering node could notify all interested
nodes directly.  However, such an algorithm is both more complicated due to race conditions, and requires $O(N^2)$
active messages.  As we will see in Section~\ref{subsec:eventmicro}, the latency of a single event trigger
active message is very small, and any algorithm that is super-linear with the node count will not scale well.
%either is or will soon be an scalability issue for large systems.

The data structures used to track an event cannot be freed until all operations on that event have been
performed.  Creation and triggering can each happen only once, but there can be an arbitrary number of operations
that are dependent on an event.   Furthermore, some operations may not be requested until long after the
event has triggered.  Other systems incorporating events address this by reference counting event
handles\cite{Khronos:OpenCL}, but such reference counting adds both programmer and runtime overhead even when
limited to a single node; further overhead and complexity would be expected for a cluster-level
reference-counting approach.

%Instead of focusing on freeing event data structures, our implementation aggressively recycles them, 
%needing fewer total event data structures than a referencing counting implementation while also eliminating
%the overhead of reference counting.  The key observation is that one {\em generational event} data structure
% can efficiently capture the state of one {\em untriggered} event and a very large number (e.g. $2^{32}-1$)
%of already-triggered events.  An event's handle is expanded to include its generation number as well as the
%identifier for the underlying generational event.  In addition to the triggered-or-not state and list of
%dependent operations
%for the most recent generation, the owning node also remembers how many previous generations have already
%been triggered.  (There is no need to keep a list of dependent operations on previous generations.  Any
%new operation that is dependent on a generation that is known to be already triggered can immediately be
%executed.)  When an event is created, any generational event for whom the most recent generation has 
%triggered can be reused - the generation is increased by one and the state is returned to ``untriggered.''
%As before, this can be done with no inter-node communication.
Instead of attempting to free event data structures, our implementation aggressively recycles them.
Compared to a reference counting scheme our implementation requires fewer total event data structures
and has no programmer/runtime overhead.  The key observations is that one {\em generational event}
data structure can efficiently capture the state of one untriggered event and a very large 
number (e.g. $2^{32}-1$) of already-triggered events.  To accomplish this, we extend each event
handle to include its generation number as well as the identifier for its generational event.  Each
generational event remembers how many previous generations have already been triggered.  Any
new operation that is dependent on a generation that is known to already be triggered can immediately
be executed.  A generational event can be reused for a new generation as soon as the current generation has triggered. To
create a new event, a node finds a generational event in the triggered state,
increases the generation by one, and sets the generational event's state to untriggered.  As before, 
this can be done with no inter-node communication.

%A remote node's data structure is also efficient - the boolean values for whether the event
%has triggered and whether the event has already been subscribed to are replaced with the numbers of the most
%recent generation known to have triggered (this is received in the event trigger active message) and of the
%most recent generation the remote node has subscribed to (this is sent in the event subscription active
%message).  And as with the owner node's data structure, only
%one list of dependent operations is needed.  Although the latencies of a large system can delay the
%reception of an event trigger active message, if an operation is created that depends on a later generation
%of an event than what the current list of events is dependent on, that serves as a roundabout indication that
%the previous generation has indeed triggered, allowing the existing list of operations to be executed.
Nodes also maintain generational event data structures corresponding to remote events that they have
observed.  These data structures maintain the most recent known generation to have triggered as well
as the generation of the most recent subscription message sent (if any).  The distributed nature
of the system allows remote generational events to perform an interesting optimization.  If at any point a
remote generational event receives a request to wait on a later generation than its current
generation,
then it is safe to infer that all generations up to the requested generation have already 
been triggered because a new generation of the event was already created by the event's owner node.
All local waiters can then be notified even before receiving the event trigger message for the current generation.

Barriers are implemented as an extension of events.  In place of an event's triggered/untriggered state,
a barrier's owner tracks the number of outstanding arrivals.  When the arrival count 
reaches zero, the barrier is considered to have triggered.  Remote nodes send {\em barrier update}
active messages containing an adjustment value that is positive for {\tt alter\_arrival\_count} operations 
and negative for {\tt arrive} operations.  A simplified version of vector clocks\cite{Fidge1998} is used
to detect the race condition in which the barrier owner receives an {\tt arrive} operation before the
corresponding {\tt alter\_arrival\_count}.  When detected, the decrement of the arrival count 
is delayed to avoid a spurious triggering of the barrier.

%% Events are created on demand by each node.  The ID of the creating node is encoded
%% in the upper bits of each event's ID and the creating node is said to be the 
%% event's {\em owner}.  If an event is used as an argument to an
%% operation on a remote node then the remote node can determine the owner of 
%% the event and send an active message to become a {\em subscriber} of the event.  When a
%% remote node becomes a subscriber to an event it guarantees that it will receive an active
%% message from the event's owner when the event triggers.  To minimize the amount
%% of communication that occurs between nodes, each node remembers the events to which it
%% is subscribed and guarantees that it only subscribes to an event once.  In the case
%% where there may be many remote waiters on an event 
%% this can dramatically reduce the number of inter-node active messages.

%% When an event triggers, it automatically notifies all of the operations that are waiting
%% on its local node.  The owner node will also send active messages to all of the subscriber nodes
%% telling them that the event has triggered.  Even though the event has now notified all of
%% its waiters, the memory on the owning node for storing the event cannot be reclaimed 
%% because there still may be handles to the event in the system which will request to
%% use this event in the future.  These requests will need to be satisfied saying that 
%% the event has already triggered for the duration of the application.  Rather than waste
%% the resources, we recycle event implementations.

%% A {\em dynamic event} is a logical construct that will only be triggered once and is the level
%% of abstraction at which the programmer reasons about events.  A {\em physical event} is an
%% actual event implementation that consumes resources on its owner node.  Multiple dynamic events
%% can be mapped to a single physical event.  However, for each physical event there can be at most
%% one {\em active} event in the set of dynamic events that are mapped to it.  An active event is a 
%% dynamic event that is yet to trigger.  By only having one active dyanmic event at a time 
%% there is no need to disambiguate event triggers that are sent to the physical event.  Each
%% physical event maintains a count of the number of trigger operations that it has observed.

%% When the runtime needs to return a handle corresponding to a new dynamic event, it finds
%% a physical event which currently has no active events (e.g. the trigger count equals the number
%% of dynamic events mapped to the physical event).  The runtime increases the count
%% of the number of dynamic events mapped to the physical event.  The handle that is returned 
%% corresponds to a new dynamic event which contains an ID that refers to the physical
%% event it is mapped to and a {\em generation} that is the dynamic event number for the particular
%% physical event.  Note that by definition the generation will always be one greater than the
%% trigger count at the time of dynamic event creation.

%% Within this framework it is now very easy to test whether the dynamic event specified by a
%% handle has triggered.  Using the ID that is contained within the handle we find the
%% physical event implementation (which may require an active message if the owner node is
%% remote from where the test is begin performed).  We can then compare the generation of
%% the handle to the number of observed triggers contained by the physical event.  If the
%% number of physical triggers is greater than or equal to the handle's generation, then
%% the dynamic event has triggered.  We show in Section~\ref{sec:apps} that in practice
%% the number of needed physical events is significantly less than the number of dynamic events.

%% We also leverage the mapping of dynamic events onto physical events to improve the efficiency
%% of subscriptions.  Nodes cache the last generation for every physical event for which they
%% have seen a trigger notification.  If the trigger test described above detects a trigger based
%% on the cached information then there would be no need for the subscription which would save an
%% active message.  If the event has not been detected to have
%% triggered locally then an active message is sent to the owner which would always have been necesary
%% without the dynamic to physical event mapping.

\subsection{Lock Implementation}
\label{subsec:lockimpl}

Like events, locks are created on demand by the node on which the creation request was made.  The lock
handles are also statically divided across the nodes, allowing the creation to be done without inter-node
communication.  Unlike events whose ownership is static, locks may migrate; the {\em creating}
node is the initial owner of a lock, but that ownership can be transferred to other nodes.

Locks are often used to mediate access to data, and the size of that data is often small.  To
support this common case we allow locks to be created with an associated
%avoid adding the latency of inter-node data copies to critical sections, a lock may also be created with a
{\em payload}, a small (i.e. less than 4KB - the limit for an active message payload) 
piece of data that is moved around with the lock and guaranteed to be coherent while 
the lock is held.

Since any node may at some point be the owner of a lock, all nodes use the same data structure to track the
state of a lock on the node.
%(they differ only in when the structure is allocated -
%nodes other than the creator still wait until the first reference to the lock on that node).  
The structure tracks the following:
\begin{itemize} \itemsep1pt \parskip0pt \parsep0pt
\item {\em owner node} - the most recently known owner of the lock.  If the current node is the owner, this
information is always accurate.  If not, this information may be stale, but by induction the recorded 
owner is guaranteed to know the actual owner or the next node to query about ownership.
\item {\em lock status} - records whether the lock is currently held.  This is only valid if the node
is the current owner.
\item {\em local waiters} - a list of pending local lock requests.  This data is always valid on all nodes.
\item {\em remote waiters} - a bitmask of which other nodes have pending requests.  This bitmask is only
valid on the current owner.
\item {\em local payload pointer and size} - the local node's copy of the lock payload
\end{itemize}

When a lock request is made, an event is created to track when the grant occurs.  The current node then
examines its copy of the lock data structure to determine if it is the owner.
%(creating it if this the first operation for that lock on that node).  
If the current node is the owner and the lock isn't
held, the lock is granted immediately and the event is triggered.  If the lock is held, the event is added
to the list of local waiters.  Note that the event associated with
the lock request is the only data that must be stored.
%(This is the only thing that is stored.  The actual ``requestor'' of the lock
%is implicitly captured in which operations are dependent on the lock grant event.)  
If the current node isn't the owner, a {\em lock request} active message is
sent to the most recently known owner.  If the receiving node is no
longer the owner, it will forward the message on to the next node known
to be the owner of the lock.
%(If that information proves to be stale, the request is forwarded
%by the receiving node, eventually catching up to the current owner.)  
If the current owner's status shows
the lock is currently held, the bit for the requesting node is set in remote waiters bitmask.  If the lock is
not held, the ownership of the lock is given to the requesting node via a {\em lock transfer} active
message.  A lock transfer message includes the bitmask of remote waiters and an up-to-date copy 
of the lock's payload.

Similarly, an unlock request is first checked against the local node's lock state.  If the local node is
not the owner, an {\em unlock} active message is sent to the most recently known owner, which is forwarded
if necessary.  Once the unlock request is on the lock's current owning node, the local waiter list is
examined.  If the list is non-empty, the lock remains in the locked state and the first lock grant
event is pulled off the local waiter list and triggered.  If instead the local waiter list is empty, the
lock state is changed to unlocked, and the bitmask of remote waiters is examined.  If there are any, then
one of them is chosen and the corresponding node becomes the new owner via a lock transfer active message.

The unfairness that results from a lock favoring local waiters over remote waiters is intentional.  When the
contention on a lock is high (the only time when the question of fairness is relevant), the latencies
involved in transferring a lock between nodes can be the limiter on throughput.  By minimizing the number
of lock transfers, the throughput on the lock is maximized.  This effect will be demonstrated in
Section~\ref{subsec:lockmicro}.

A similar issue can arise with the forwarding of lock request active messages.  If two nodes are
transferring a lock very quickly back and forth, a third node's request could be forwarded an arbitrary
number of times as it chases the lock around.  Although this can result in unfairness, it doesn't 
prevent forward progress.  A lock is only transferred to a node with at least one local waiter, so if 
the third node's request ever became the only one left, the lock transferring would stop and the request
would eventually catch up.

\subsection{Reduction Instances}
\label{subsec:reducimpl}

In addition to normal physical regions, our implementation supports two different layouts of data for
reduction-only physical regions called {\em reduction instances}.   
These layouts are designed to optimize for the common pattern where 
multiple tasks make commutative {\em reductions} into shared data.  Instead of having to copy the existing 
physical region to a suitable memory and then applying individual reduction operations, reduction instances 
allow a large number of reduction operations to be accumulated and transferred efficiently to the original 
location of the data. This decoupling from the original location of the data can allow allow greater parallelism
in cases where processors cannot all access the same memory (e.g. many GPUs).

The first kind of reduction-only physical region is a {\em reduction fold instance}, which may only be
used for reduction operations that support a {\em fold} operation (see Listing~\ref{lst:regionapi}).  
A reduction fold instance is similar
to a normal physical region in that it is implemented as an array, indexed by the same element indices used
for the normal physical regions.  The difference is that each array element is a value of the reduction's
{\tt RHS} type which is generally the same size or smaller than the actual element type of the region 
(i.e. the {\tt LHS} type).  When the client requests a reduction operation to a particular location, 
the supplied right-hand-side value is folded in
to the corresponding location in the array.  An arbitrary number of reductions can be done to each location.
When the reduction fold instance is copied back to a normal instance, the contents of the reduction instance
are transferred in a large block to the location of the normal instance.  On the receiving side
the runtime automatically applies the reduction instance to the destination region
by invoking the reduction function once for each location in the region via a cache-friendly 
linear sweep over the memory.

The second kind of reduction instance is a {\em reduction list instance}.  Although it generally doesn't
perform as well as a reduction fold instance, it can be better for cases with sparse updates.  It is also
valid option for when a reduction operation doesn't support a {\em fold} operation.  Instead of folding
together multiple reductions to the same location, a reduction list instance keeps a log of every
reduction operation requested (i.e. the location and the right-hand-side value).  When the reduction list 
instance is copied back to a normal physical region, the whole list is transferred and then replayed at
the target region's location.  Except in sparse update cases, this list is often larger than the target
region, and any randomness in the sequence of locations in the list can impact the memory access efficiency
of the reductions to the target region.  However, as we will show in section~\ref{subsec:reducmicro}, 
the benefits of enabling parallelism and hiding latency usually more than make up for these drawbacks.

