
\section{Programming Interface}
\label{sec:interface}
The target set of clients for our interface are higher-level language
compilers and runtimes such as Legion\cite{Legion12} as well as 
advanced systems programmers.  This class of users demand total control
over the underlying hardware and transparent performance from an
implementation.  To support these demands we have designed our interface
to be as low-level as possible.  In many cases we have chosen to trade-off
programmability for performance.  We eschewed any features that would
automatically be performed without being directed by the client.
Our goal in the design of this interface is to provide a set of low-level,
high-performance primitives that are asynchronous and composable.

Our interface organizes functionality into objects.  We describe
these objects in five parts: events for composing operations 
(Section~\ref{subsec:events}), processors for parallel computation 
(\ref{subsec:procs}), deferred locks for synchronization
(\ref{subsec:locks}), physical regions for data layout and movement
(\ref{subsec:phyreg}), and a machine object for introspection of the underlying
hardware (\ref{subsec:machmodel}).  In all cases (except for the
machine object which is a static singleton), instances of the objects
are light-weight {\em handles} that provide a unique name for an underlying
implementation.  Multiple copies of handles are permitted and handles can be 
passed by value.  Furthermore, every handle is valid everywhere in the system.  
This property allows a client to pass handles by value between computations 
without having to reason about the distributed nature of the system.
%We describe how we support this property in more detail in 
%Section~\ref{sec:impl}.

\lstset{
  captionpos=b,
  language=C++,
  basicstyle=\scriptsize,
  numbers=left,
  numberstyle=\tiny,
  columns=fullflexible,
  stepnumber=1,
  escapechar=\#,
  keepspaces=true,
  %literate={<}{{$\langle$}}1 {>}{{$\rangle$}}1,
  %morekeywords={region,coloring,partition,spawn,disjoint,aliased},
  %deletekeywords=float,
}

\subsection{Events}
\label{subsec:events}
Events are the primary mechanism for describing dependences in our system.
Listing~\ref{lst:eventapi} shows the interface for events.  An instance of the {\tt Event} type 
names a unique event in the system.  {\tt NO\_EVENT} (line 5) is a special instance
of an event that by definition has always triggered.  The event interface
supports testing whether an event has triggered (line 7) and waiting on
an event to trigger (line 9).  While these methods can be useful, the encouraged
use of events is passing them as preconditions for other operations.  The client can use the
{\tt merge\_events} call (lines 11-12) to create an event corresponding to the 
conjunction of a set of events.

In most cases events are created as the result of other
operations and the implementation is responsible for triggering these events.  Users 
can create a special type of event called a {\tt UserEvent} (line 15) that the
user has the power to trigger.  A user event is a sub-type of an event
that has a {\tt trigger} method (line 18).  Users can also create another
type of event called a {\tt Barrier} that will require multiple arrivals before
the event is triggered.  The user can manage the number of arrivals that must
be seen as well as how many arrivals occur at a time.  Once a barrier's arrival
count goes to zero, the barrier will trigger.  Note that barriers in our interface
provide a superset of the functionality of traditional barriers since they can
be used in a blocking manner (via the {\tt wait} call), but are primarily used
asynchronously.  Lastly, since user events and barriers are sub-types of 
an event they can be used wherever an event is required.

\begin{lstlisting}[float={t},label={lst:eventapi},caption={Event Interface.}]
class Event {
public:
  const unsigned id;
  const unsigned gen;
  static const Event NO_EVENT;
  // check if an event has triggered
  bool has_triggered() const;
  // wait on the event
  void wait() const;
  // Merge events together to create a new event
  static Event merge_events(Event ev1, Event ev2);
  static Event merge_events(const set<Event> &to_merge);
};

class UserEvent : public Event {
public:
  static UserEvent create_user_event();
  void trigger() const;
};

class Barrier : public Event {
public:
  static Barrier create_barrier(unsigned expected_arrivals);
  void alter_arrival_count(int delta) const;
  void arrive(unsigned count = 1) const;
};
\end{lstlisting}

\subsection{Processors}
\label{subsec:procs}
Listing~\ref{lst:procapi} shows the interface for {\tt Processor} objects which allow for
the creation of parallel computations.  Processor provide a way of naming 
every computational unit in the machine.  We describe how to discover processor handles
in Section~\ref{subsec:machmodel}.  In the case of our current interface 
processors are either individual cores or discrete GPUs, but this is easily modified 
to include new processor types (line 10).   Processors support a single {\tt spawn}
operation (lines 13-14) that will launch a new {\em task} on that processor.
Note that because the spawn operation
is invoked on a processor handle, it is possible to launch a task on any
processor from anywhere in the system.  
The spawn call takes an optional event that must trigger before the task can begin.  
The spawn call returns an event that will trigger when the task completes.

Lines 17-23 of Listing~\ref{lst:procapi} show a simple example using 
the processor and event interface.  The {\tt diamond\_task} launches four
sub-tasks with a diamond dependence pattern.  The {\tt merge\_events} call
is used to merge the events {\tt eb} and {\tt ec} to create the shared dependence
for the last sub-task.

\begin{lstlisting}[float={t},label={lst:procapi},caption={Processor Interface and Example.}]
class Processor {
public:
  const unsigned id;
  typedef unsigned TaskFuncID;
  typedef void (*TaskFuncPtr)
            (const void *args,size_t arglen,Processor p);
  typedef map<TaskFuncID, TaskFuncPtr> TaskIDTable;

  enum Kind {
    CPU_PROC,GPU_PROC,// ... future processor types
  };

  Event spawn(TaskFuncID func_id,const void *args,size_t arglen,
              Event wait_on = Event::NO_EVENT) const;
};

void diamond_task(const void *args,size_t arglen,Processor p){
  Event ea = p.spawn(TASK_A,NULL,0);
  Event eb = p.spawn(TASK_B,NULL,0,ea);
  Event ec = p.spawn(TASK_C,NULL,0,ea);
  Event em = Event::merge_events(eb,ec);
  Event ed = p.spawn(TASK_D,NULL,0,em);
}
\end{lstlisting}

\subsection{Deferred Locks}
\label{subsec:locks}

Events and barriers express ordering properties between operations, but in many
cases ordering is too strict of a requirement.  For many applications access to data must only be atomic and
not necessarily ordered.  Traditionally locks have been used to serialize access to
data.  However, all lock implementations that we are aware of require either blocking
or spinning, neither of which composes well with asynchronous operations.
Deferred locks are a new synchronization mechanism that allows for synchronization
in an completely asynchronous environment.  

Listing~\ref{lst:lockapi} shows the 
interface for locks.  Unlike blocking locks, the {\tt lock} method (line 4) doesn't
block but instead always returns immediately with an event that will be triggered
when the lock has been acquired.  Just like the spawn method, the lock method also 
takes an optional event parameter as a precondition.  Similarly, the {\tt unlock} 
method (line 5) takes an optional event precondition parameter.

An important difference between deferred locks and blocking locks is that the processor
that requests the lock doesn't have to be the one that uses it.  A common
convention in writing code with deferred locks is to acquire locks on behalf of a task being
launched.  Listing~\ref{lst:lockapi} illustrates this with a simple example.  The
{\tt launcher\_task} is calling a sub-task that is going to require the {\tt needed}
lock.  A lock request is issued and the resulting {\tt lock\_event} is used as
a precondition for launching the {\tt SUB} task.  The unlock operation is then
precondition on the event corresponding to {\tt SUB} task's completion.  The
{\tt SUB} task can run safely while holding the lock and the lock is released when the
{\tt SUB} task completes.  Using deferred locks in this style prevents 
compute resources from waiting on locks by ensuring tasks only run when their needed
locks have already been acquired.

Between deffered locks and barriers (described in Section~\ref{subsec:events}) clients 
have access to the same set of synchronization primitves that they traditionally have
in both threading and bulk-synchronous interfaces.  However, these operations can now 
be composed with other asynchronous operations which is not possible in any other interface
of which we are aware.

%Deferred locks provide a super-set of the functionality of blocking locks.  As their
%name suggests, deferred locks can use the event corresponding to their lock acquire
%operation to defer execution until the lock acquire has been granted.  Deferred
%locks can also be converted back into a blocking lock by immediately waiting on the event
%returned from a call to {\tt lock}.  In addition, deffered locks also provide {\tt mode} and
%and {\tt exclusive} parameters that allow the user to specify whether other requests can acquire
%the lock simultaneously.  A lock can only be acquired in one mode at a time.  The
%{\tt exclusive} parameter specifies whether other owners are permitted once the
%current lock request is granted.


\begin{lstlisting}[float={t},label={lst:lockapi},caption={Deferred Lock Interface and Example.}]
class Lock {
public:
  const unsigned id;
  Event lock(unsigned mode = 0, bool exclusive, 
              Event wait_on = Event::NO_EVENT) const;
  void unlock(Event wait_on = Event::NO_EVENT) const;
  // Create a new lock, destroy an existing lock
  static Lock create_lock();
  void destroy_lock();
};

void launcher_task(const void *args, size_t arglen, Processor p) {
  ...
  // Unpack lock from arguments
  Lock needed = ...
  // Acquire lock
  Event lock_event = needed.lock();
  // Launch sub-task
  Event task_event = p.spawn(SUB,NULL,0,lock_event);
  // Release lock
  needed.unlock(task_event);
  ...
}
\end{lstlisting}

\subsection{Physical Regions}
\label{subsec:phyreg}
In distributed machines with discrete memories, data movement often consists of more
than a simple copy operation.  Many applications perform operations and transformations on
their data in conjunction with data movement for higher performance.  One common example is 
that applications accumulate reduction operations in a buffer and as part of a copy operation 
the reduction operations are applied to a destination buffer on the receiving side of the copy.  
To support these kinds of conjoined operations in an asynchronous environment, our interface 
must be aware of the structure of data so it can perform accompanying operations with 
data movement operations.  To describe the structure of data our interface uses {\em physical regions.}

A physical region is an allocation of data in a single memory in the memory hierarchy.  Physical
regions are grouped into {\em classes} which are sets of physical regions that share the
same number and size of elements, but don't necessarily share the same data layout.  Our
interface supports sub-classing, but the details are omitted due to space constraints.  The
interface only supports data movement operations between physical regions that are of
the same class (or between super- and sub-classes).  Note, clients can still pass arbitrary arrays
of bits via the processor {\tt spawn} call.
% Say something about dynamically modifying the number of elements in a class

Listing~\ref{lst:regionapi} shows a subset of the interface for physical regions.  The
first object in the interface is a {\tt RegionMetaData} object (line 1).  Physical
regions created by the same region meta-data object belong to the same class.  Physical
regions are represented by {\tt RegionInstance} objects (line 17).  When creating a physical
region the client must specify the {\tt Memory} in which the physical region is going
to be allocated.  Memories are described in more detail in Section~\ref{subsec:machmodel}.
If there is not enough space for the physical region allocation a {\tt NO\_INST} will
be returned.  There is no automatic coherence of data between physical regions in the
same class.  It is the client's responsiblity to manage data coherence using copies
between physical regions (line 22-23).  Just like other operations copies can be
predicated on an event and return an event corresponding to completion.

Clients can also associate an operation as part of a physical region.  One example of
associating an operation with a physical region is shown on lines 13-14.  This instance
of the {\tt create\_instance} method
will create a physical region called a {\em reduction instance} that will be assoicated 
with the given {\tt ReductionOp} specified by the ID.  The interface for a {\tt ReductionOp} is shown on 
lines 26-33.  A reduction operation must have an {\tt apply} method, and can optionally 
implement two additional methods, {\tt init} and {\tt fold} that support additional 
optimizations described in Section~\ref{subsec:reducimpl}.  The two template parameters 
describe the base element type of the region ({\tt LHS}:left-hand-side) and the type of 
the argument to the reduction ({\tt RHS}:right-hand-side).  The reduction operation has 
total control over the choice of data layout for the reduction instance (not shown).
Reduction instances can be copied to other reduction instances with the same operation
or to regular physical regions of the same class.

An example of using reduction instances is shown on lines 35-49 of Listing~\ref{lst:regionapi}.
In this example, two reduction instances, {\tt inst1} and {\tt inst2}, are created in different memories to
be used with reduction operation 2 (lines 40-41).  A physical region of the same class {\tt inst3}
is also created in a different memory (line 42).  Two tasks are launched that will apply
reductions locally into their respective reduction buffers (lines 43-44).  The resulting
events are then used to chain copies back to {\tt inst3} after the tasks complete (45-46).  As
part of these copies the reductions stored in {\tt inst1} and {\tt inst2} will be applied
to {\tt inst3}.  Since reductions are done atomically, there is no race between the reduction
applications.  Finally, a merge on the events from the movement and reduction applications is 
performed and a final task using {\tt inst3} is launched (lines 47-48).

Reductions and reduction instances illustrate only a single case where physical regions provide 
the structure necessary to conjoin an operation with data movement.  Physcal regions make our
interface easily extensible to arbitrary transformations and operations as a part of data movment.
While physical regions are a slightly higher-level construct than an annonymous
buffer of bits, we believe that the optimizations they enable are important in an asynchronous
environment.  Clients also still have access to an interface for controlling data layout through
the operations associated with physical regions which mitigates the loss in generality.

%{\em Physical regions} are the mechanism for reasoning about
%the placement and movement of data in the interface.  Physical regions are an 
%allocation of data in a single memory in the memory hierarchy.  Listing~\ref{lst:regionapi} 
%shows a subset of the physical region interface.  The first object in the interface
%is a {\tt RegionMetaData} object (line 1).  Region meta-data objects provide the interface for the
%creation of sets of physical regions that possess the same number and type of elements (but not necessarily
%the same data layout).  The interface is only able
%to copy data between two physical regions that were created by the
%same region meta-data object.  A runtime error will be generated if a copy is attempted
%between two physical regions that were not created by the same meta-data object.

%An instance of a physical region is represented by a {\tt RegionInstance} object (line 17).
%When creating a physical region the programmer must specify the {\tt Memory}
%in which the physical region is going to be allocated (line 10).  Memories are described
%in more detail in Section~\ref{subsec:machmodel}.  Once the physical region is created it
%cannot be moved.   There is no coherence of data between physical regions created by the 
%same meta-data object.  It is the programmer's responsiblity to manage data coherence
%using copies between physical regions (line 22-23).  Just like other operations,
%copies can be predicated on an event and return an event corresponding to
%completion.  
%A copy between two physical regions can only be performed if copies
%are permitted between the two memories where the regions were created.  
%The legality of copies can be discovered using queries to the {\tt Machine} object
%described in Section~\ref{subsec:machmodel}.

%The data held inside a physical region is accessed by another kind of object called an
%{\em accessor} (not shown).  Accessors support read, write, and reduction operations
%to elements inside the physical region.  Accessors can be specialized for the case where
%all memory is known to be directly accessible to the executing processor, which allows 
%for direct references to elements.  In cases where not all memory is directly accessable, 
%accessors supply the level of indirection necessary to convert operations into remote 
%memory operations (RMAs) if supported by the underlying hardware.  
%An example of this is described in more detail in Section~\ref{sec:impl}.

%One special (but common) case for applications is when reductions need to be performed.
%For reductions it is usually more efficient to store reduction
%operations in a local reduction buffer and then merge reduction buffers together at a later
%point in time.  To support this case the interface allows for the creation of a special kind
%of physical instance called a {\em reduction instance}.  The call to {\tt create\_instance} on
%lines 13-14 takes a {\em reduction operation} to be associated with a physical instance which
%tells the interface to create a reduction instance.  Reduction instances can only be accessed
%using a reduction of the given reduction type; any other accesses will result in a runtime
%error.  A reduction operation is a functor which must have at least one method of the type
%$(T_1 \rightarrow T_2 \rightarrow T_1)$ which takes a region element of type $T_1$ and a value
%to be reduced of type $T_2$ and creates and new region element.  Reduction operations can optionally 
%support two additional methods: a method that supplies an initial value of type $T_2$ and a 
%fold method of type $(T_2 \rightarrow T_2 \rightarrow T_2)$.  The presence of these methods
%enable additional optimizations for reductions described in Section~\ref{subsec:reducimpl}.

%Reduction instances can be copied using the same interface as regular physical instances.
%It is only legal to copy a reduction instance to a regular physical instance or another
%reduction instance of the same reduction operator.  Any attempt to copy from a regular physical 
%instance to a reduction instance will result in a runtime error.  A copy from a reduction 
%instance to a regular physical instance will apply all the reductions in the reduction 
%buffer to the physical instance.  If a copy is from one reduction instance to another, 
%the two reduction buffers will be merged.  The semantics of a merge are dependent upon 
%the underlying implementation of reduction buffers which is discussed in more detail 
%in Section~\ref{subsec:reducimpl}.

\begin{lstlisting}[float={t},label={lst:regionapi},caption={Physical Region Interface and Example.}]
class RegionMetaData {
public:
  const unsigned id;
  static const RegionMetaData NO_REGION;
  // Create and destroy metadatas
  static RegionMetaData create_region(size_t num_elmts, 
                                      size_t elmt_size);
  void destroy_region() const;
  // Create and destroy instances
  RegionInstance create_instance(Memory memory) const;
  void destroy_instance(RegionInstance instance) const;
  // Create a physical instance only for reductions
  RegionInstance create_instance(Memory memory, 
                        ReductionOpID redopid) const;
};

class RegionInstance {
public:
  const unsigned id;
  static const RegionInstance NO_INST;
  // Copy between instances
  Event copy_to(RegionInstance target, 
                Event wait_on = Event::NO_EVENT);
};

template<typename LHS, typename RHS>
class ReductionOp {
public:
  void apply(LHS *lhs, RHS rhs);
  // Optional methods for optimizations
  void init(RHS *rhs, size_t num_elmts);
  void fold(RHS *rhs1, RHS rhs);
};

void reduction_ex(const void *args,size_t arglen,Processor p) {
  Memory m1,m2,m3; Processor p1,p2,p3; ReductionOpID redop = 2;
  ... // Choose memory and processors
  RegionMetaData meta = RegionMetaData::
                        create_region(num_elmts,elmt_size);
  Instance inst1 = meta.create_instance(m1,redop);
  Instance inst2 = meta.create_instance(m2,redop);
  Instance inst3 = meta.create_instance(m3);
  Event t1 = p1.spawn(REDUC,&inst1,sizeof(RegionInstance));
  Event t2 = p2.spawn(REDUC,&inst2,sizeof(RegionInstance));
  Event c1 = inst1.copy_to(inst3,t1);
  Event c2 = inst2.copy_to(inst3,t2);
  Event em = Event::merge_events(c1,c2);
  Event t3 = p3.spawn(USE,&inst3,sizeof(RegionInstance),em);
}
\end{lstlisting}


\subsection{Machine}
\label{subsec:machmodel}
The last component of the interface allows for introspection of the current machine.
Listing~\ref{lst:machineapi} shows the machine interface.
The {\tt Machine} object is a singleton object that serves two purposes.  First, it
is the mechanism for initializing the runtime; at the start of a program the client 
creates a machine object, instantiates a task table mapping task IDs to function pointers,
and invokes the {\tt run} method (line 30-31).
Second, the machine object provides an interface for the client to 
discover properties of the underlying machine (lines 35-43).  The machine object maintains
sets of all the unique {\tt Processor} and {\tt Memory} handles in the system.  Processors
were described in Section~\ref{subsec:procs}.  For every memory in the system there also 
exists a {\tt Memory} handle with a unique ID.  Different memories often, but not always, 
imply a different address space.  Examples of different memories are described in
Section~\ref{sec:impl}.

%For
%example, in a large cluster each node would have a different {\tt Memory} handle
%corresponding to each node's DRAM memory.
%However, if the cluster supported multiple GPUs then there would be a different
%{\tt Memory} handle for each GPU's {\em zero-copy} memory which shares part of the 
%address space with the corresponding node memory.  Memories will be used for describing
%the placement of data described in Section~\ref{subsec:phyreg}.

%The machine interface 
%contains three different object types: {\tt Processor} (line 1), {\tt Memory} (line 17), and a 
%singleton {\tt Machine} object (line 22).  For every processor (i.e. CPU core or GPU) in 
%the system there exists a {\tt Processor} handle with a unique ID naming that processor.  
%{\tt Processor} objects support a single {\tt spawn} operation (line 13-14) that will
%launch a {\em task} on that processor.   

The machine object also maintains two relations between the set of processors
and the set of memories:
\begin{itemize}
\item Processor-Memory: for each processor-memory pair whether
the processor can directly access (read and write) the memory and 
latency and bandwidth properties if access is possible
\item Memory-Memory: for every pair of memories whether copies can be
performed and latency and bandwidth properties if copies are possible
\end{itemize}

The client can query these two relations by the {\tt get\_*\_affinity} calls
(lines 40-43) to discover properties of the machine.  

%These calls populate a vector with affinity structures (not shown)
%that contain latency and bandwidth information.  The programmer can restrict the query
%by specifying a specific processor or memory, otherwise information is returned
%about all processors and memories in the system.  

\begin{lstlisting}[float={t},label={lst:machineapi},caption={Machine Interface.}]

class Memory {
public:
  const unsigned id;
};

class Machine {
public:
  Machine(int *argc, char ***argv,
          const Processor::TaskIDTable &task_table);
  enum RunStyle {
    ONE_TASK_ONLY,ONE_TASK_PER_NODE,
    ONE_TASK_PER_PROCESSOR,
  };
  void run(Processor::TaskFuncID task_id = 0, 
        RunStyle = ONE_TASK_ONLY, const void *args, size_t arglen);
public:
  static Machine* get_machine(void); // pointer to Machine
  // References to all processors and memories
  const set<Memory>& get_all_memories(void) const;
  const set<Processor>& get_all_processors(void) const;
  Processor::Kind get_processor_kind(Processor p) const;
  size_t get_memory_size(Memory m) const;
  // Query properties of underlying hardware
  int get_proc_mem_affinity(vector<ProcMemAffinity &result,
          Processor restrict_proc = 0, Memory restrict_mem = 0);
  int get_mem_mem_affinity(vector<MemMemAffinity> &result,
          Memory restrict_mem1 = 0, Memory restrict_mem2 = 0);
};
\end{lstlisting}


