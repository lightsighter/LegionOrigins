
\section{Related Work}
\label{sec:related}

MPI is the current industry standard for programming large
supercomputers\cite{MPI}.  While MPI supports asynchronous
communication operations there is no mechanism for composing
asynchronous operations.  GASNet is another interface for programming
large clusters that centers around active messages\cite{GASNET07}.  GASNet
is a part of the heterogeneous implementation of our interface, but like
MPI, does not support composition of asynchronous active messages with
other constructs.  Co-array Fortran, UPC, and Titanium
are array based languages that implement bulk-synchronous
programming models similar to MPI that allow for asynchronous
exchanges of data\cite{COARRAY_FORTRAN,UPC99,JV:Yel98}.  Like
MPI however, none of these languages allow for general composition
of asynchronous tasks, communication, and synchronization.

Both POSIX threads and OpenMP\cite{OPENMP98} are used for intra-node parallel
programming in large machines, but neither support any asynchronous
operations.  CUDA\cite{CUDA} and OpenCL\cite{Khronos:OpenCL} support
the composition of asynchronous kernel launches and asynchronous copies
between a host node and a single GPU.  However, the only synchronization
options available in both interfaces are blocking operations on either
a stream or a work queue.  OpenCL has events similar to our interface for
expressing ordering, but has no synchronization primitive for expressing
more relaxed properties such as atomicity that can be described by a
deferred lock.  OpenCL events are not valid globally, but can only be
used within a single GPU context.

Cilk is a parallel programming model that has demonstrated the power of
asynchronous function calls\cite{CILK95}.  The Cilk {\em work-first}
principle provides a compelling theoretical argument for parallel programming
interfaces such as ours that optimize for throughput at the potential
expense of adding additional latency to the critical 
dependence path\cite{Frigo98}.

The Threaded Abstract Machine (TAM) is a programming model designed to
make it easy to port dataflow programs onto a parallel machine\cite{CullerGSvE93}.  
Conceptually our interface is similar to the dataflow languages TAM supported.
However, the implementation of TAM is designed for a class of much smaller machines
and therefore is based on heavier communication and threading mechanisms.
While TAM supports composable asynchronous computation and communication, its
synchronization mechanisms still require blocking operations.

Chapel\cite{Chamberlain:Chapel} and X10\cite{X1005} are high-level parallel
programming languages that support asynchronous operations.  The constructs
introduced in these languages are higher-level and have complex 
semantics.  We view our interface as providing a target
for implementing these higher-level asynchronous operations in an efficient
manner similar to Legion\cite{Legion12}.

Physical regions in our interface are related to
arrays from the Sequoia runtime interface\cite{Houston08} in that they
both allow the runtime to do specialized operations on data by being
aware of the structure of data.  Sequoia's runtime interface supported
asynchronous tasks and copies, but did not permit composition of
asynchronous operations.

The implementation of our interface shares similarities with large
distributed systems.  Many distributed systems implement a publish/subscribe
abstraction for supporting communication that is similar in some ways to how
we manage events and event waiters\cite{Aguilera99,Carzaniga01}.  Work has
also been done on using object oriented languages to build event-based
distributed systems\cite{Eugster01,Harrison97,Chang91}.  In these cases
callbacks are registered to run when event operations are triggered
remotely.  Events in these systems are much heavier weight
and often carry large data payloads.  In many distributed event systems
the focus is on resiliency instead of performance\cite{Ostrowski09}.
