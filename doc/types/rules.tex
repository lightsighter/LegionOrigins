
\documentclass{article}

\usepackage{fancyhdr}
\usepackage{datetime}

\lhead{DRAFT - \today - \currenttime}

\usepackage{fullpage}
\usepackage[cmex10]{amsmath}
\usepackage{listings}

\newcommand{\infrule}[2]{\displaystyle\frac{\displaystyle\strut{#1}}{\displaystyle\strut {#2}}}
\newcommand{\cinfrule}[3]{\parbox{14cm}{\hfil$\infrule{#1}{#2}$\hfil}\parbox{4cm}{$\,#3$\hfil}}

\newcommand{\regionexpand}[0]{[r'_1/r_1, \ldots, r'_n/r_n]}
\newcommand{\rtriple}[3]{\left({#1},{#2},{#3}\right)}
\newcommand{\rsingle}[1]{\rtriple{#1}{\emptyset}{\top}}
\newcommand{\rtripsub}[3]{\rtriple{#1}{\Phi_{#2}}{C_{#3}}}
\newcommand{\envsub}[2]{\Gamma, \Phi_{#1}, C_{#2} \vdash}

\begin{document}

\pagestyle{fancy}

\begin{table*}
\centering
{\small
\begin{tabular}{ccc}
$program$ & ::= & $fn^*$ \\
$fn$ & ::= & $f(x_1, \ldots, x_n)$ $e$ \\
\end{tabular}
}
\caption{Program and Functions}
\end{table*}


\begin{table*}
\centering
{\small
\begin{tabular}{cc}
%
% Statement composition
%
\begin{math}
\infrule{
\begin{array}{lc}
  \envsub{1}{1} e_1 : \rtripsub{T_1}{2}{2} \\
  \envsub{2}{2} e_2 : \rtripsub{T_2}{3}{3} 
\end{array}
}
{
  \envsub{1}{1} e_1; e_2 : \rtripsub{T_2}{3}{3}
}
\end{math} 
&\raisebox{-0.2in}{[{\tt Composition}]} \\

%
% Conditional statement
%
\begin{math}
\infrule{
\begin{array}{lc}
  \envsub{1}{1} e_1 : \rtripsub{\mathit{Int}}{2}{2} \\
  \envsub{2}{2} e_2 : \rtripsub{T}{3}{3} \\
  \envsub{2}{2} e_3 : \rtripsub{T}{4}{4} \\
  C_3 \models C_5 \hspace{1cm} C_4 \models C_5
\end{array}
}
{
  \envsub{1}{1}\textrm{ if }e_1\textrm{ then }e_2\textrm{ else }e_3 : \rtriple{T}{\left(\Phi_3 \cap \Phi_4\right)}{C_5}
}
\end{math}
&\raisebox{-0.2in}{[{\tt Conditional}]} \\
%
% While Statement
%
\begin{math}
\infrule{
\begin{array}{lc}
  \Phi_2 \subset \Phi_1 \\
  C_1 \models C_2 \\
  \envsub{2}{2} e_1 : \rtripsub{\mathit{Int}}{3}{3} \\
  \envsub{3}{3} e_2 : \rtripsub{T}{2}{2}
\end{array}
}
{
  \envsub{1}{1}\textrm{ while }e_1\textrm{ do }e_2 : \rtripsub{T}{3}{3}
% TODO: fix this - while can't return something of type T if the initial condition test fails!
}
\end{math}
&\raisebox{-0.2in}{[{\tt While}]} \\
%
% Assignment statement
%
\begin{math}
\infrule
{
\begin{array}{lc}
  \{ x : T \} \in \Gamma \\
  \envsub{1}{1} e_1 : \rtripsub{T}{2}{2}
\end{array}
}
{
  \envsub{1}{1} x = e_1 : \rtripsub{T}{2}{2}
}
\end{math}
&\raisebox{-0.2in}{[{\tt Assignment}]} \\
\end{tabular}
}
\caption{Statements}
\end{table*}

\begin{table*}
\centering
{\small
\begin{tabular}{cc}

%
% Variables
%
\begin{math}
\infrule
{
  \{ x : T \} \in \Gamma
}
{
  \envsub{1}{1} x : \rtripsub{T}{1}{1}
}
\end{math} 
&\raisebox{-0.2in}{[{\tt Variable}]} \\

%
% Integers
%
\begin{math}
\infrule
{
}
{
  \envsub{1}{1} i : \rtripsub{\mathit{Int}}{1}{1}
}
\end{math} 
&\raisebox{-0.2in}{[{\tt Integer Literal}]} \\

%
% Tuples
%
\begin{math}
\infrule
{
\begin{array}{lc}
  \envsub{1}{1} e_1 : \rtripsub{T_1}{2}{2} \\
  \envsub{2}{2} e_2 : \rtripsub{T_2}{3}{3} \\
  \ldots \\
  \envsub{n}{n} e_n : \rtripsub{T_n}{n+1}{n+1}
\end{array}
}
{
  \envsub{1}{1} \langle e_1, \ldots, e_n \rangle : \rtripsub{\langle T_1, \ldots, T_n \rangle}{n+1}{n+1}
}
\end{math}
&\raisebox{-0.2in}{[{\tt Tuple}]}  \\

%
% Field Access
%
\begin{math}
\infrule
{
  \envsub{1}{1} e : \rtripsub{\langle T_1, \ldots, T_n \rangle}{2}{2}
}
{
  \envsub{1}{1} e.i : \rtripsub{T_i}{2}{2}
}
\end{math} 
&\raisebox{-0.2in}{[{\tt Field Access}]} \\

%
% Lambda
%
\begin{math}
\infrule
{
\begin{array}{lc}
  FIXME \\
        \Gamma, x : T_1 \vdash e : T_2 / C
\end{array}
}
{
        \Gamma \vdash \lambda x.e : T_1 \rightarrow T_2 / C, \emptyset
}
\end{math}
&\raisebox{-0.2in}{[{\tt Lambda}]} \\

%
% Function application
%
\begin{math}
\infrule
{
\begin{array}{lc}
  \envsub{1}{1} e_1 : \rtripsub{\rtripsub{T_i}{i}{i} \rightarrow \exists r_1, \ldots, r_n . \rtripsub{T_o}{o}{o}}{2}{2} \\
  \envsub{2}{2} e_2 : \rtripsub{T_i}{3}{3} \\
  r'_1, \ldots, r'_n \not\in \Gamma \hspace{1cm} 
  \Phi_i \subset \Phi_3 \hspace{1cm}
  C_3 \models C_i
\end{array}
}
{
  \envsub{1}{1} e_1 e_2 : \rtriple{\regionexpand T_o}{\left(\Phi_3 \cup \regionexpand \Phi_o\right)}{\left(C_3 \wedge \regionexpand C_o\right)}
}
\end{math}
&\raisebox{-0.2in}{[{\tt Function Application}]} \\ 

%
% Pack
%
\begin{math}
\infrule
{
\begin{array}{lc}
  RR = \exists r_1, \ldots, r_n.\rtriple{T_1}{\emptyset}{C_1} \\
  \envsub{2}{2} e : \rtripsub{\regionexpand T_1}{3}{3} \\
  C_3 \models \regionexpand C_1
\end{array}
}
{
  \envsub{2}{2}\textrm{ packrr }RR\mbox{ }e : \rtripsub{RR}{3}{3}
}
\end{math}
&\raisebox{-0.2in}{[{\tt Pack}]} \\

%
% Unpack
%
\begin{math}
\infrule
{
\begin{array}{lc}
  RR = \exists r_1, \ldots, r_n.\rtriple{T_1}{\emptyset}{C_1} \\
  \envsub{2}{2} e : \rtripsub{RR}{3}{3} \\
  r'_1, \ldots, r'_n \not\in \Gamma
\end{array}
}
{
  \envsub{2}{2}\textrm{ unpackrr }e : \rtriple{\regionexpand T_1}{\Phi_3}{\left(C_3 \wedge \regionexpand C_1\right)}
}
\end{math}
&\raisebox{-0.2in}{[{\tt Unpack}]}

\end{tabular}
}
\caption{Expressions}
\end{table*}

%%%%%%%%%%%%%%%%%
% Operations on Region Elements
%%%%%%%%%%%%%%%%%
\begin{table*}
\centering
{\small
\begin{tabular}{cc}
%
% Read Region
% 
\begin{math}
\begin{array}{l}
read : \forall T,r_1,r_2. \rtriple{\langle Region(r_1,T), T@r_2 \rangle}{\{ readable(r_1) \}}{(r_2 \le r_1)} \rightarrow \rsingle{T}
\end{array}
%read : \forall T,r_1,r_2. \langle Region(r_1,T), T@r_2 \rangle \rightarrow T / (readable(r_1) \wedge r_2 \le r_1)
\end{math} & [{\tt Read Pointer}] \\

%
% Write Region
%
\begin{math}
write : \forall T,r_1,r_2. \rtriple{\langle Region(r_1,T), T@r_2, T \rangle}{\{ writeable(r_1) \}}{(r_2 \le r_1)} \rightarrow \rsingle{T}
%write : \forall T,r_1,r_2. \langle Region(r_1,T), T@r_2, T \rangle \rightarrow () / (writeable(r_1) \wedge r_2 \le r_1)
\end{math} & [{\tt Write Pointer}] \\

%
% Reduce Region
%
\begin{math}
\begin{array}{l@{}l}
reduce : \forall T,r_1,r_2. \big( & \forall f:(T \rightarrow T \rightarrow T). \\
& \rtriple{\langle Region(r_1,T), T@r_2, f, T \rangle}{\{ reduceable(r_1,f) \}}{(r_2 \le r_1)} \rightarrow \rsingle{()} \big) \end{array}
%reduce : \forall T,r_1,r_2. \big( \forall f:(T \rightarrow T \rightarrow T).  \langle Region(r_1,T), T@r_2, f, T \rangle \rightarrow () / (reduceable(r_1,f) \wedge r_2 \le r_1) \big)
\end{math} & [{\tt Reduce Pointer}] \\

%
% Alloc 
%
\begin{math}
alloc : \forall T,r. \rtriple{\langle Region(r,T) \rangle}{\{ allocable(r) \}}{\top} \rightarrow \rsingle{T@r}
%alloc : \forall T,r. \langle Region(r,T) \rangle \rightarrow T@r / allocable(r)
\end{math} & [{\tt Alloc Pointer}] \\

%
% Free
%
\begin{math}
free : \forall T,r_1,r_2. \rtriple{\langle Region(r_1,T), T@r_2 \rangle}{\{ freeable(r_1) \}}{(r_2 \le r_1)} \rightarrow \rsingle{()}
%free : \forall T,r. \langle Region(r,T), T@r \rangle \rightarrow () / freeable(r)
\end{math} & [{\tt Free Pointer}] 

\end{tabular}
}
\caption{Predefined Functions on Region Elements}
\end{table*}

%%%%%%%%%%%%%%%%%%%
% Operations on Regions
%%%%%%%%%%%%%%%%%%%
\begin{table*}
\centering
{\small
\begin{tabular}{cc}

%
% Partition
%
\begin{math}
partition : \forall T,r. \rtriple{\langle Region(r,T), (T@r \rightarrow \mathit{Int}) \rangle}{\{readable(r)\}}{\top} \rightarrow \rsingle{Partition(r)}
\end{math} & [{\tt Partition}] \\

%
% Subregion
%
\begin{math}
\begin{array}{lc}
\begin{array}{l@{}l}
subregion : \forall T,r. & \rtriple{\langle Partition(r), \langle i_1, \ldots, i_n \rangle \rangle}{\emptyset}{\bigwedge_{j,k = 1, \ldots, n, j \ne k} i_j \ne i_k}
\rightarrow \\
& \exists r_1, \ldots, r_n. \rtriple{\langle Region(r_1,T), \ldots, Region(r_n,T) \rangle}{\emptyset}{\bigwedge_{j,k = 1, \ldots, n, j \ne k}. r_j * r_k}
\end{array}
\end{array}
\end{math} & [{\tt Subregion}] \\

%
% Smash
%
\begin{math}
\begin{array}{l@{}l}
smash : \forall T,r_1,\ldots,r_n,r_p. & \rtriple{\langle Region(r_1,T), \ldots, Region(r_n,T) \rangle}{\emptyset}{(\bigwedge_{i = 1, \ldots, n} r_i \le r_p)} \rightarrow \\
& \exists r'.  \rtriple{Region(r\prime,T)}{\emptyset}{(r\prime \le r_p) \wedge (\bigwedge_{i = 1, \ldots, n} r_i \le r')}
\end{array}
\end{math} & [{\tt Smash}] \\

\end{tabular}
}
\caption{Predefined Functions on Regions}
\end{table*}

%%%%%%%%%%%%%%%%%%%%
% Operations on Region Relationships
%%%%%%%%%%%%%%%%%%%%
\begin{table*}
\centering
{\small
\begin{tabular}{cc}

%
% New
%
\begin{math}
\infrule{
  RR = \exists r_1, \ldots, r_n. \rtriple{T_1}{\emptyset}{C_1}
}{
  newrr RR : \rsingle{()} \rightarrow \exists r_1, \ldots, r_n. \rtriple{T_1}{\bigcup_{i = 1..n} \Big( readable(r_i) \cup writeable(r_i) \cup allocable(r_i) \cup freeable(r_i) \Big)}{C_1}
}
%newRR : () \rightarrow RR
\end{math} & [{\tt New RR}] \\

%
% Delete
%
\begin{math}
deleteRR : \forall RR. \rsingle{RR} \rightarrow \rsingle{()}
\end{math} & [{\tt Delete RR}] \\
\end{tabular}
}
\caption{Predefined Functions on Region Relationships}
\end{table*}

%%%%%%%%%%%%%%%%%%
% Types
%%%%%%%%%%%%%%%%%%
\begin{table*}
\centering
{\small
\begin{tabular}{ccc}

T & ::= & Int \\
  &$\mid$&$\langle T_1, \ldots, T_n \rangle$ \\
  &$\mid$&$\rtripsub{T_i}{i}{i} \rightarrow \exists r_1, \ldots, r_n.\rtripsub{T_o}{o}{o}$ \\
  &$\mid$&$T@r$\\
  &$\mid$&$\forall t.T$ \\
  &$\mid$&$\forall r.T/C$ \\
  &$\mid$&$RR$ \\
  &$\mid$&$Region(r,T)$ \\
  &$\mid$&$Partition(r)$ \\

\end{tabular}
}
\caption{Types}
\end{table*}

%%%%%%%%%%%%%%%%%%
% Region Relationship Types
%%%%%%%%%%%%%%%%%%
\begin{table*}
\centering
{\small
\begin{tabular}{ccc}
RR & ::= & $\exists r_1, \ldots, r_n.\rtriple{T}{\emptyset}{C}$ \\
\end{tabular}
}
\caption{Region Relationship Types}
\end{table*}



%%%%%%%%%%%%%%%%%%%%
% Constraints
%%%%%%%%%%%%%%%%%%%%
\begin{table*}
\centering
{\small
\begin{tabular}{ccc}

C & ::= & $(r_1 \le r_2)$ \\
  &$\mid$&$(r_1 * r_2)$ \\
  &$\mid$&$(i_1 = i_2)$ \\
  &$\mid$&$(i_1 \ne i_2)$ \\
  &$\mid$&$(C_1 \wedge C_2)$ \\
\end{tabular}
}
\caption{Constraints}
\end{table*}


%%%%%%%%%%%%%%%%%%%%
% Saturation of Constraints
%%%%%%%%%%%%%%%%%%%%
\begin{table*}
\centering
{\small
\begin{tabular}{cc}
%
% reflexivity/transitivity of subregion relationship
%
\begin{math}
\begin{array}{c}
\forall r, r \le r \\
\forall r_1, r_2, r_3. (r_1 \le r_2 \wedge r_2 \le r_3) \rightarrow r_1 \le r_3
\end{array}
\end{math} & [{\tt Subregion Relationships}] \\
\\
\begin{math}
\forall r_1, r_2, r_3. (r_1 \le r_2 \wedge r_2 * r_3) \rightarrow r_1 * r_3
\end{math} & [{\tt Subregion Disjointness}] \\
\\
\begin{math}
\forall r_1, r_2. r_1 * r_2 \rightarrow r_2 * r_1
\end{math} & [{\tt Disjointness Symmetry}]
\end{tabular}
}
\caption{Constraint Inference Rules}
\end{table*}


%%%%%%%%%%%%%%%%%%%%
% Priviledges
%%%%%%%%%%%%%%%%%%%%
\begin{table*}
\centering
{\small
\begin{tabular}{ccc}

$\phi$ & ::= & $readable(r)$ \\
  &$\mid$&$writeable(r)$ \\
  &$\mid$&$reduceable(r,f)$ \\
  &$\mid$&$allocable(r)$ \\
  &$\mid$&$freeable(r)$ \\
\\
$\Phi$ & ::= & $\{ \phi_1, \ldots, \phi_n \}$
\end{tabular}
}
\caption{Region Privileges}
\end{table*}

\begin{table*}
\centering
{\small
\begin{tabular}{cc}
%
% Subregion Readability
%
\begin{math}
\infrule{
\begin{array}{lc}
  \envsub{1}{1} e_1 : \rtripsub{T_1}{2}{2} \\
  readable(r_1) \in \Phi_2 \hspace{1cm} C_2 \models r_2 \le r_1
\end{array}
}
{
  \envsub{1}{1} e_1 : \rtriple{T_2}{\Phi_2 \cup \{ readable(r_2) \}}{C_2}
}
\end{math} 
&\raisebox{-0.2in}{[{\tt Subregion Readability}]} \\
%
% Subregion Writeability
%
\begin{math}
\infrule{
\begin{array}{lc}
  \envsub{1}{1} e_1 : \rtripsub{T_1}{2}{2} \\
  writeable(r_1) \in \Phi_2 \hspace{1cm} C_2 \models r_2 \le r_1
\end{array}
}
{
  \envsub{1}{1} e_1 : \rtriple{T_2}{\Phi_2 \cup \{ writeable(r_2) \}}{C_2}
}
\end{math} 
&\raisebox{-0.2in}{[{\tt Subregion Writeability}]} \\
%
% Subregion Reduceability
%
\begin{math}
\infrule{
\begin{array}{lc}
  \envsub{1}{1} e_1 : \rtripsub{T_1}{2}{2} \\
  reduceable(r_1) \in \Phi_2 \hspace{1cm} C_2 \models r_2 \le r_1
\end{array}
}
{
  \envsub{1}{1} e_1 : \rtriple{T_2}{\Phi_2 \cup \{ reduceable(r_2) \}}{C_2}
}
\end{math} 
&\raisebox{-0.2in}{[{\tt Subregion Reduceability}]} \\
%
% Subregion Allocability
%
\begin{math}
\infrule{
\begin{array}{lc}
  \envsub{1}{1} e_1 : \rtripsub{T_1}{2}{2} \\
  allocable(r_1) \in \Phi_2 \hspace{1cm} C_2 \models r_2 \le r_1
\end{array}
}
{
  \envsub{1}{1} e_1 : \rtriple{T_2}{\Phi_2 \cup \{ allocable(r_2) \}}{C_2}
}
\end{math} 
&\raisebox{-0.2in}{[{\tt Subregion Allocability}]} \\
%
% Subregion Freeability
%
\begin{math}
\infrule{
\begin{array}{lc}
  \envsub{1}{1} e_1 : \rtripsub{T_1}{2}{2} \\
  freeable(r_1) \in \Phi_2 \hspace{1cm} C_2 \models r_2 \le r_1
\end{array}
}
{
  \envsub{1}{1} e_1 : \rtriple{T_2}{\Phi_2 \cup \{ freeable(r_2) \}}{C_2}
}
\end{math} 
&\raisebox{-0.2in}{[{\tt Subregion Freeability}]} \\
%
% Reduction Incompatibility
%
\begin{math}
\infrule{
\begin{array}{lc}
  \envsub{1}{1} e_1 : \rtripsub{T_1}{2}{2} \\
  reduceable(r, f_1) \in \Phi_2 \hspace{1cm} reduceable(r, f_2) \in \Phi_2
\end{array}
}
{
  \envsub{1}{1} e_1 : \rtriple{T_2}{\Phi_2 \cup \{ readable(r), writeable(r) \}}{C_2}
}
\end{math} 
&\raisebox{-0.2in}{[{\tt Reduction Incompatibility}]} \\
\begin{math}
\infrule{
\begin{array}{lc}
  \envsub{1}{1} e_1 : \rtripsub{T_1}{2}{2} \\
  readable(r) \in \Phi_2 \hspace{1cm} writeable(r) \in \Phi_2
\end{array}
}
{
  \envsub{1}{1} e_1 : \rtriple{T_2}{\Phi_2 \cup \{ reduceable(r, f) \}}{C_2}
}
\end{math} 
\end{tabular}
}
\caption{Privilege Inference}
\end{table*}


\end{document}
