
\section{Related Work}

The Chapel programming language \cite{Chamberlain:Chapel} is the closest related work to the Legion
programming model.  Chapel has several concepts built into the language to aid in
the expression of locality.  The first concept is that of domains.  Domains
are similar to logical regions in that they describe a collections of indexes
into items of the same type.  Domains can create sub-domains by slicing the index sets from
a parent domain.  There are two differences between domains and logical regions.
First, domains are a higher level concept than regions since they track index sets which
have concepts such as dimensionality and support iterators.  Logical regions in Legion 
are only allowed to be accessed with pointers.  Second, the act of creating
subdomains doesn't track disjointness information the way that partitioning of logical regions
does making it challenging for the compiler or the runtime to infer task independence.

In addition to domains, Chapel also supports the notion of domain maps and locales to enable 
the programmer to efficiently map domains onto hardware \cite{CHAPEL11}.  Locales are a flat array
of abstract locations.  Programmers can use locales by writing domain maps that specify
how domains are broken up and assigned to locales.  Domain maps provide the same functionality
as partitions and mappers in Legion, but require the user to correctly implement domain
maps in order for the program to be correct.  Legion explicitly isolates correctness
from performance by defining the Mapper interface.  In addition, Chapel's flat array of locales
makes it challenging to fully utilize deep memory hierarchies.  Chapel currently supports
clusters and GPUs in isolation \cite{CHAPELGPU}, but does not report results that 
make use of both.

Sequoia is an example of a locality-aware programming language capable of 
expressing computations in deep memory hierarchies \cite{Fatahalian06}.  
Sequoia is a special case
of the Legion progamming model in which only arrays can be recursively partitioned.
The tree of array partitions is also required to exactly match the shape of the task tree.  
Sequoia requires the programmer to statically map arrays and task trees 
using a machine specific mapping file.
Legion generalizes the Sequoia model by allowing for partitioning of pointer data structures through 
regions, decoupling the region tree from the task tree, and enabling dynamic 
mappings through the mapper interface.  Sequoia is capable of running on
combinations of many machines including MPI clusters of Cell processors, but
is unable to use GPUs.

X10 is another parallel programming environment designed to operate on distributed
memory machines \cite{X1005}.  In X10 the concept of places enables programmers to talk about where
to place both data and tasks in the machine.  However, once data and tasks have
been placed they are fixed which mandates that data movement be explictly managed by
user level code or implicitly by the compiler or runtime \cite{X1008}.  Recently X10 has introduced
the concept of a region into the compiler's intermediate representation \cite{X1011}.
Unlike Legion, regions in X10 are not visible to the programmer but are inferred
from high level arrays through static analysis.  X10 provides support for clusters of GPUs
\cite{X10GPU}, but requires the programmer to write all code managing data movement
through both the cluster and GPU memory hierarchies.

SPMD languages such as Titanium \cite{TIT98} and UPC \cite{UPC99} have mechanisms for
describing array partitions in distributed memories.  However, the partition
operations supported only operate on two-level memory hierarchies consisting of local
and global memory and do not reason about pointer data structures.  Neither Titanium
nor UPC currently support GPUs as part of their language standard \cite{TITANIUMSTANDARD}
\cite{UPCSTANDARD}.  Part of Legion's 
low-level runtime system is constructed using UPC's underlying GASNet 
runtime system \cite{GASNET07}. 

In previous programming systems regions have primarily been used as a construct for
describing memory management schemes \cite{REAPS02}\cite{RC01}  
or for enforcing security policies \cite{CYCLONE01}.  In all cases, regions had
memory layout implications.  Logical regions in Legion enable the programmer to
describe locality and independence without any memory layout implications.

Marino introduces a generic flow-insensitive type-effect system capable of verifying 
accesses to statically tagged locations of memory \cite{PRIVLIGES09}.  Legion's programming
system allows for dynamic creation and partitioning of regions requiring a 
flow-sensitive analysis to track privleges.
