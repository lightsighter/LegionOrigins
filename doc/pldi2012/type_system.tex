\section{Type System}
\label{sec:type}

The key to both performance and correctness in Legion is the
accuracy of the region usage declarations.  The Legion type system has been
designed to statically check that region usage declarations are adhered to.
In this section we summarize the main features of the type system; due to space limitations we omit more than
we include.
Type judgments have the form:
\begin{center}
$\Gamma, \Phi_i, C_i \vdash e : T, \Phi_o, C_o$
\end{center}
In addition to tracking the types of variables $\Gamma$, the
type environment includes the current privileges $\Phi_i$ 
and region constraints $C_i$.  A judgment concludes $e$ has some type $T$,
but also tells us how the evaluation of $e$ changes the privileges $\Phi_o$ and constraints $C_o$.

Table~\ref{tbl:priv_const} shows the form of the individual privileges and
constraints that make up $\Phi$ and $C$.  Privileges represent the ability to
perform some operation on a region and its subregions.  Constraints capture relationships between regions
resulting from partitioning: $a \rleq b$ if $a$ is a subregion $b$ (directly or transitively) and $a \ast b$ if
$a$ and $b$ are disjoint (e.g., because they are distinct subregions of a disjoint partition).
Our first type rule checks region reads:

\begin{center}
{\small
\begin{math}
\infrule{
\begin{array}{lc}
  \multicolumn{2}{c}{\Gamma, \Phi_1, C_1 \vdash p : T@r_1, \Phi_2, C_2} \\
  C_2 \models r_1 \le r_2 & readable(r_2) \in \Phi_2
%  \envsub{1}{1} e_1 : \rtripsub{T_1}{2}{2} \\
%  \envsub{2}{2} e_2 : \rtripsub{T_2}{3}{3} 
\end{array}
}
{
  \Gamma, \Phi_1, C_1 \vdash read(r_2, p) : T, \Phi_2, C_2
%  \envsub{1}{1} e_1; e_2 : \rtripsub{T_2}{3}{3}
}
\end{math} 
}
\end{center}

In order to safely read from a pointer into $r_1$, we must 
provide a region $r_2$ which is known to contain $r_1$ and for which we possess
a read privilege.  Rules for region writes and reductions are similar.

Region relationships are first-class values and can be stored in data structures
and passed around the program.  However, the Legion type
system cannot afford to lose information about regions that are stored
in the heap.  To track the regions
in region relationships accurately we use existential quantification
over regions.  The quantifiers capture both types and constraints, but 
not privileges.  The type of a region relation has the form:
\begin{center}
$RR = \exists r_1, \ldots, r_n.\left(T, \emptyset, C\right)$
\end{center}
Pack and unpack operations are used in the type system
to add and remove the existential quantifiers:
\begin{center}
{\small
\begin{math}
\begin{array}{c}
\infrule{
\begin{array}{l}
RR = \exists r_1, \ldots, r_n.\left(T_1, \emptyset, C_1\right) \\
\Gamma, \Phi_2, C_2 \vdash e : RR, \Phi_3, C_3  \\
r'_1, \ldots, r'_n \not\in \mathit{RegionsOf}\left(\Gamma, \Phi_3, C_3\right)
\end{array}
}{
\begin{array}{l@{}l}
\Gamma, \Phi_2, C_2 \vdash unpackrr{~}e :~  & [r'_1/r_1,\ldots,r'_n/r_n]T_1, \Phi_3, \\
& C_3 \wedge [r'_1/r_1,\ldots,r'_n/r_n]C_1
\end{array}
}
\\
\\
\infrule{
\begin{array}{l}
RR = \exists r_1, \ldots, r_n.\left(T_1, \emptyset, C_1\right) \\
\Gamma, \Phi_2, C_2 \vdash e : [r'_1/r_1,\ldots,r'_n/r_n]T_1, \Phi_3, C_3 \\
C_3 \models [r'_1/r_1,\ldots,r'_n/r_n]C_1
\end{array}
}{
\begin{array}{l}
\Gamma, \Phi_2, C_2 \vdash packrr{ }RR\mbox{ }e : RR, \Phi_3, C_3 \\
\end{array}
}
\end{array}
\end{math}
}
\end{center}
Any expression whose type is a region relationship can be unpacked.
Doing so introduces a fresh region variable into the current type
environment for each region that was bound in the quantification.  The
fresh region variables are substituted into the region relationship's
type and constraints.  Packing an expression into a region
relationship does the reverse.  If the type of an expression can be
unified with the region relationship's type and the corresponding
constraints of the region relationship can be shown to hold, the
expression can be packed into the region relationship and used
interchangeably with any other instance of that region relationship.

Although the type system uses explicit packing and unpacking operations, 
these are implicit in Legion application code, such as the
circuit simulation in Listing~\ref{lst:code_ex}.  Unpacking is
inserted whenever a variable comes into scope, and packing is used
whenever a variable is written to the heap or passed as a task argument.

%Additionally, the simultaneous assignment operation essentially
%performs a pack for the result of the assignments and will also perform an
%unpack if one ore more fields of the structure are left unchanged.

\begin{table}
\centering
{\small
\begin{math}
\begin{array}{cc}
\begin{tabular}{ccc}
$\phi$ & ::= & $readable(r)$ \\
  &$\mid$&$writeable(r)$ \\
  &$\mid$&$reduceable(r,f)$ \\
%  &$\mid$&$allocable(r)$ \\
%  &$\mid$&$freeable(r)$ \\
\\
$\Phi$ & ::= & $\{ \phi_1, \ldots, \phi_n \}$
\end{tabular}
&
\begin{array}{ccc}
C & ::= & (r_1 \rleq r_2) \\
%  &\mid&(r_1 \le r_2) \\
  &\mid&(r_1 * r_2) \\
%  &\mid&(i_1 = i_2) \\
%  &\mid&(i_1 \ne i_2) \\
  &\mid&(C_1 \wedge C_2) \\
\end{array}
\end{array}
\end{math}
}
\label{tbl:priv_const}
\caption{Privileges and Constraints}
\end{table}

As mentioned above, region privileges are not stored in packed types;
instead, privileges are passed through the task call tree.  This is a key
property of the type system, and is what allows the runtime to 
track which regions are in use by which tasks and to make distributed
scheduling decisions.
Each subtask must obtain its privileges from
the task that calls it.  For example, in Listing~\ref{lst:code_ex},
the task {\tt calc\_new\_currents} requires regions {\tt piece.rw\_pvt} with {\tt RW} privileges
and {\tt piece.rn\_pvt} and {\tt piece.rn\_ghost} with {\tt RO} (read-only) privileges.  Because the calling task has {\tt RW} privileges on the parent regions of these
region arguments, the type checker is able to confirm that the subtask can execute with the needed privileges.

%The coherence property for a task argument is always identical to the one of the parent task.

Due to the inability for a packed region relationship to hold privileges, any
useful region relationship will leave at least one region free (i.e. unquantified) and include constraints that relate the quantified region(s) to the free one(s).  As long as a task holds a privilege for the region that is kept free, the
region relationship can be unpacked and used.



%bprivilege for a given region is the \emph{newrr} operation that created that
%region, but the way in which privileges can be provided to a subtask, or
%returned to a parent task, is described in the task application rule:
%
%\begin{center}
%\begin{math}
%\begin{array}{c}
%\infrule
%{
%\begin{array}{lc}
%  \Gamma, \Phi_1, C_1 \vdash e_1 : \left(\left(T_i, \Phi_i, C_i\right) \rightarrow \exists r_1, \ldots, r_n . \left(T_o, \Phi_o, C_o\right), \Phi_2, C_2\right) \\
%  \Gamma, \Phi_2, C_2 \vdash e_2 : \left(T_i, \Phi_3, C_3\right) \\
%  r'_1, \ldots, r'_n \not\in \mathit{RegionsOf}(\Gamma, \Phi_3, C_3) \hspace{1cm} 
%  \Phi_i \subset \Phi_3 \hspace{1cm}
%  C_3 \models C_i
%\end{array}
%}
%{
%\begin{array}{l@{}l}
%  \envsub{1}{1} e_1 e_2 : \rtriple{& \regionexpand T_o}{\left(\Phi_3 \cup \regionexpand \Phi_o\right)}{\\ & \left(C_3 \wedge \regionexpand C_o \wedge \bigwedge_{\substack{{\footnotesize r_f \in \{ r | \regionexpand C_o \models fresh(r) \} } \\ {\tiny r_o \in \mathit{RegionsOf}(\Gamma, \Phi_3, C_3)}}} r_f * r_o\right)}
%  \Gamma, \Phi_1, C_1 \vdash e_1 e_2 : \Big( & [r'_1/r_1,\ldots,r'_n/r_n]T_o, \\
% & \left(\Phi_3 \cup [r'_1/r_1,\ldots,r'_n/r_n] \Phi_o\right) , \\
% & \left(C_3 \wedge [r'_1/r_1,\ldots,r'_n/r_n] C_o\right) \Big)
%\end{array}
%}
%\end{array}
%\end{math}
%\end{center}
%
%In this rule, expression $e_1$ has the type of a Legion task.  In addition
%to the task's input and output types, it specifies the privileges and
%constraints that must exist at the point where the task is called.  It also
%describes the privileges and constraints that exist as post-conditions of the
%task invocation.  The output type, privileges, and constraints can be
%captured in an existential quantification, allowing a task to return one
%or more regions that were previously unknown to the caller.
