\section{Mapping}
\label{sec:mapping}

While the Legion programming model describes locality and independence information abstractly,
to run a Legion application on a machine we must make concrete decisions about
where tasks will be run and where region instances will be placed.  Instead of placing
the burden of decision making on the programming system, we introduce a programmable mapping
interface that gives a programmer total control over how these decisions are made.  Armed
with this interface, the programmer can then specify application- or machine-specific
mapping decisions that would be difficult for a general-purpose programming system to infer.
In this section we describe the general mapping interface (Section \ref{sec:mapinterface}),
our base implmentation of this interface (Section \ref{sec:defmapper}), and the benefits
of creating custom mappers (Section \ref{sec:custommap}).

\subsection{Mapping Interface}
\label{sec:mapinterface}
A mapper is any object that implements the mapping interface.  The mapping interface
consists of ten function calls that the runtime will make to request direction from
the programmer about how a program should be executed.  To aid in performing the 
decisions made in these calls the mapper has access to a simple 
interface for inspecting the properties of the machine on which it is running.  This includes 
information about all the processors and their kinds (i.e. CPU,GPU), as well as the memories visible
to each processor and their latencies and bandwidths.  For brevity we only
discuss three of the mapping interface calls.

\begin{itemize}
\item {\tt select\_initial\_processor} - for each task the runtime system will
ask the mapper to which processor the task should be sent.  The mapper 
can choose to keep the task on the local processor or send it to any other processor
in the system.

\item {\tt permit\_task\_steal} - when handling a steal request from another processor
the runtime will ask the mapper to decide which tasks are allowed to be stolen.  
By always returning an empty set, the mapper can disable stealing.

\item {\tt map\_task\_region} - for each logical region requested for a task, the
runtime will query the mapper for a prioritized list of memories where a physical instance for this
task should be placed.  To aid the mapping decision the runtime provides a list of the current valid physical intances
of the logical region.  The mapper returns a list of memories in which the runtime
should attempt to either reuse or manifest a physical instance of the logical region.  The
runtime traverses this list and searches for a currently valid instance.  If none can be
found, it attempts to allocate a physical instance and then issue the necessary copies
to retrive the valid data.  If both options fail, it moves on to the next memory in the list.
\end{itemize}

There are two very elegant properties of this mapping interface.  The first property is
that all decisions made by a mapper are orthogonal to the correctness of the program and can only 
impact performance.  Regardless of where a mapper chooses to place a task or map a region, 
the runtime will always schedule tasks and copies in accordance with the privileges and 
coherence properties specified in the Legion program.  Therefore when writing an application
in Legion, a programmer can begin by using the default mapper and later optimize his program
by creating a custom mapper and gradually refining its mapping decisions.

The second useful property of the mapping interface is that it isolates machine-specific decisions
to a specific module of code.  As a result, Legion programs are highly
portable.  To port a Legion program to a new architecture, a programmer only has to
implement a new mapper with decisions specific to the new architecture. 

\subsection{Default Mapper}
\label{sec:defmapper}
To make writing Legion applications easier, we provide a default mapper implementation that
can be used to get an application working quickly at a moderate performance level.  The 
default mapper employs a simple scheme for mapping tasks and regions.  When
{\tt select\_initial\_processor} is invoked for the default mapper, it first checks for which kind of processors
the task has variants (i.e. GPU).  If the fastest variant is for the kind of processor the mapper is managing
it will choose to keep the task local, otherwise it will send the task to the closest processor of the
fastest variant kind.

The default mapper employs a Cilk-like algorithm for task stealing\cite{CILK95}.  Tasks are kept local
to their target processor whenever possible and only moved when stolen.  Unlike Cilk, the default mapper
has the information necessary for locality-aware stealing.  When {\tt permit\_task\_steal} is called for a task, 
the default mapper inspects the logical regions for the task being stolen, and will mark that other tasks using 
the same logical regions should be stolen as well.

For calls to {\tt map\_task\_region}, the default mapper constructs a stack of memories ordered from best-to-worst
by bandwidth from the local processor.  This stack is then returned as the location of memories to be used for
mapping each region.  Note that this is a greedy algorithm that works well for the common case, but can cause some 
regions to be pulled unnecessarily close to the processor and consume precious fast memory space.  

\subsection{Custom Mappers}
\label{sec:custommap}
To optimize a Legion program or library, programmers can create one or more custom mappers.  
For each Legion task call the programmer specifies which mapper should be invoked by the runtime for
mapping that particular task.  This allows for the composition of Legion applications and libraries
each with their own custom mappers.

Each custom mapper extends the default mapper.  A programmer only has to override the functions necessary 
for creating a custom mapping.  Custom mappers can be used to create totally static mappings, 
mappings that memoize their results, or even totally dynamic mappings.  We describe examples of 
custom mappers in Section \ref{sec:exp}.




