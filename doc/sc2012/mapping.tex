\section{Mapping Interface}
\label{sec:mapping}

As mentioned previously, a novel aspect of Legion is the {\em mapping interface},
which gives programmers control over where tasks run and where region instances are placed,
making possible application- or machine-specific mapping decisions that would be difficult for a 
general-purpose programming system to infer.
We describe the interface (Section \ref{sec:mapinterface}),
our base implementation (Section \ref{sec:defmapper}), and the benefits
of creating custom mappers (Section \ref{sec:custommap}).

\subsection{The Interface}
\label{sec:mapinterface}

The mapping interface consists of ten methods that SOOPs call for
mapping decisions.  A {\em mapper} implementing these methods has access to a simple interface
for inspecting properties of the machine, including a list of
processors and their type (i.e. CPU, GPU), a list of
memories visible to each processor, and their latencies and bandwidths.
For brevity we only discuss the three most important interface calls:

\begin{itemize}
\item {\tt select\_initial\_processor} - For each task $t$ in its mapping queue a SOOP will
ask for a processor for $t$.  The mapper can keep the task on the local processor or send it to 
any other processor in the system.

\item {\tt permit\_task\_steal} - When handling a steal request a SOOP
asks which tasks may be stolen.  Stealing can be disabled by always returning the empty set.

\item {\tt map\_task\_region} - For each logical region $r$ used by a task, a
SOOP asks for a prioritized list of memories where a physical instance of $r$ should be placed.  
The SOOP provides a list of $r$'s current valid physical instances;
the mapper returns a priority list of memories in which the SOOP
should attempt to either reuse or create a physical instance of $r$.  Beginning with
the first memory, the SOOP uses a current valid instance if one is present.  Otherwise,
the SOOP attempts to allocate a physical instance and issue copies
to retrieve the valid data.  If that also fails, the SOOP moves on to the next memory in the list.
\end{itemize}

The mapping interface has two desirable properties.  
First, program correctness is unaffected by mapper decisions, which can only impact 
%First, all mapper
%decisions are orthogonal to program correctness and can only impact
performance.  Regardless of where a mapper places a task or region,
the SOOPs schedule tasks and copies in accordance with the privileges
and coherence properties specified in the program.  Therefore, when
writing a Legion application, a programmer can begin by using the
default mapper and later improve performance by creating and refining a custom mapper.
Second, the mapping interface isolates machine-specific decisions
to the mapper.  As a result, Legion programs are highly
portable.  To port a Legion program to a new architecture, a programmer need only
implement a new mapper with decisions specific to the new architecture. 

\subsection{Default Mapper}
\label{sec:defmapper}
To make writing Legion applications easier, we provide a default
mapper that can quickly get an application working
with moderate performance.  The default mapper employs a
simple scheme for mapping tasks.  When {\tt
select\_initial\_processor} is invoked, the mapper
checks the type of processors for which task $t$ has
implementations (e.g., GPU).  If the fastest implementation is for the
local processor the mapper keeps $t$ local, otherwise
it sends $t$ to the closest processor of the fastest kind that can run
$t$.


The default mapper employs a Cilk-like algorithm for task
stealing \cite{CILK95}.  Tasks are kept local 
whenever possible and only moved when stolen.  Unlike Cilk, the
default mapper has the information necessary for locality-aware
stealing.  When {\tt permit\_task\_steal} is called for a task, the
default mapper inspects the logical regions for the task being stolen
and marks that other tasks using the same logical regions should
be stolen as well.

For calls to {\tt map\_task\_region}, the default mapper constructs a stack of memories ordered from best-to-worst
by bandwidth from the local processor.  This stack is then returned as the location of memories to be used for
mapping each region.  This greedy algorithm works well in common cases, but can cause some 
regions to be pulled unnecessarily close to the processor, consuming precious fast memory.

\subsection{Custom Mappers}
\label{sec:custommap}
To optimize a Legion program or library, programmers can create one or more custom mappers.  
For each Legion task call the programmer specifies which mapper should be invoked by the runtime for
mapping that particular task.  This allows for the composition of Legion applications and libraries
each with their own custom mappers.

Each custom mapper extends the default mapper.  A programmer need only override the functions 
%necessary for creating a custom mapping.  
he wishes to customize.  Custom mappers can be used to create totally static mappings, 
mappings that memoize their results, or even totally dynamic mappings.  We describe examples of 
custom mappers in Section \ref{sec:exp}.




