\lstset{
  captionpos=b,
  language=C++,
  basicstyle=\scriptsize,
  numbers=left,
  numberstyle=\tiny,
  columns=fullflexible,
  stepnumber=1,
  escapechar=\#,
  keepspaces=true,
  literate={<}{{$\langle$}}1 {>}{{$\rangle$}}1,
  morekeywords={region,coloring,partition,spawn,disjoint,aliased},
  deletekeywords=float,
}
\begin{lstlisting}[float={t},label={lst:code_ex},caption={Circuit simulation.}]
struct Node { float voltage, new_charge, capacitance; };
struct Wire<rn> { Node@rn in_node, out_node; float current, ... ; };
struct Circuit { region   r_all_nodes; /* contains all nodes for the circuit */
                 region   r_all_wires; /* contains all circuit wires */ };
struct CircuitPiece {
  region  rn_pvt, rn_shr, rn_ghost; /* private, shared, ghost node regions */
  region  rw_pvt;                   /* private wires region */ };

void simulate_circuit(Circuit c, float dt) : RWE(c.r_all_nodes,c.r_all_wires)
{
  // The construction of the colorings is not shown.  The colorings wire_owner_map,
  // node_owner_map, and node_neighbor_map have MAX_PIECES colors 
  // 0..MAX_PIECES #$-$# 1. The coloring node_sharing map has two colors 0 and 1.
  //
  // Partition of wires into MAX_PIECES pieces
  partition<disjoint> p_wires = c.r_all_wires.partition(wire_owner_map); 
  // Partition nodes into two parts for all-private vs. all-shared
  partition<disjoint> p_nodes_pvs = c.r_all_nodes.partition(node_sharing map);

  // Partition all-private into MAX_PIECES disjoint circuit pieces
  partition<disjoint> p_pvt_nodes = p_nodes_pvs[0].partition(node_owner_map);
  // Partition all-shared into MAX_PIECES disjoint circuit pieces
  partition<disjoint> p_shr_nodes = p_nodes_pvs[1].partition(node_owner_map);
  // Partition all-shared into MAX_PIECES ghost regions, which may be aliased
  partition<aliased> p_ghost_nodes = p_nodes_pvs[1].partition(node_neighbor_map);

  CircuitPiece pieces[MAX_PIECES];
  for(i = 0; i #$<$# MAX_PIECES; i++) 
    pieces[i] = { rn_pvt: p_pvt_nodes[i], rn_shr: p_shr_nodes[i],
                  rn_ghost: p_ghost_nodes[i], rw_pvt: p_wires[i] };
  for (t = 0; t #$<$# TIME_STEPS; t++) {
    spawn (i = 0; i #$<$# MAX_PIECES; i++) calc_new_currents(pieces[i]);
    spawn (i = 0; i #$<$# MAX_PIECES; i++) distribute_charge(pieces[i], dt);
    spawn (i = 0; i #$<$# MAX_PIECES; i++) update_voltages(pieces[i]);
  }
}

void calc_new_currents(CircuitPiece piece):
        RWE(piece.rw_pvt), ROE(piece.rn_pvt,piece.rn_shr,piece.rn_ghost) {
  foreach(w : piece.rw_pvt)
    w#$\rightarrow$#current = (w#$\rightarrow$#in_node#$\rightarrow$#voltage - w#$\rightarrow$#out_node#$\rightarrow$#voltage) / w#$\rightarrow$#resistance;
}

void distribute_charge(CircuitPiece piece, float dt):
        ROE(piece.rw_pvt), RdA(piece.rn_pvt,piece.rn_shr,piece.rn_ghost) {
  foreach(w : piece.rw_pvt) {
    w#$\rightarrow$#in_node#$\rightarrow$#new_charge += -dt * w#$\rightarrow$#current;
    w#$\rightarrow$#out_node#$\rightarrow$#new_charge +=  dt * w#$\rightarrow$#current;
  }
}

void update_voltages(CircuitPiece piece): RWE(piece.rn_pvt,piece.rn_shr) {
  foreach(n : piece.rn_pvt, piece.rn_shr) {
    n#$\rightarrow$#voltage += n#$\rightarrow$#new_charge / n#$\rightarrow$#capacitance;
    n#$\rightarrow$#new_charge = 0;
  }
}
\end{lstlisting}
