
\section{Conclusion}
\label{sect:conclusion}

Modern architectures have dramatically increased in complexity in recent years and we believe that new programming models
are needed to take advantage of the potential of such machines.  We have presented the semantics and type system
for Legion, which differs from previous efforts in having a much more flexible programming model (in particular,
regions are first-class values and data can be dynamically partitioned in as many ways as needed) while maintaining
soundness of privileges and coherence.  The key insight is that we can divide the problem between static and dynamic
checking differently, keeping what is expensive to check dynamically (region pointer checks) in the static system while incurring very little if any performance cost for the work deferred to runtime (the region aliasing checks).  We have also given
a formal basis for hierarchical, distributed scheduling in Legion, and provided experimental evidence  that the
expected benefits of the Legion design are realized in real-world programs.

