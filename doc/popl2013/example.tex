
\section{Circuit Example}
\label{sec:example}

% This is a description of how the listings should be formatted.
% It can go anywhere before the listings.
\lstset{
  captionpos=b,
  language=Haskell,
  basicstyle=\scriptsize,
  numbers=left,
  numberstyle=\tiny,
  columns=fullflexible,
  stepnumber=1,
  escapechar=\#,
  keepspaces=true,
  literate={<}{{$\langle$}}1 {>}{{$\rangle$}}1,
  morekeywords={function,rr,int,float,bool,isnull,partition,as,downregion,upregion,reads,writes,rdwrs,reduces,read,write,reduce,using,unpack,pack,coloring,multicoloring,color,newcolor,atomic,simultaneous},
  deletekeywords={float,head,min,max}
}

\begin{lstlisting}[float={t},label={lst:circuit_ex},caption={Circuilt Simulation}]
-- <voltage,current,charge,capacitance>
type CircuitNode        = <float,float,float,float>
-- < owned node, owned or ghost node, resistance, current>
type CircuitWire<rn,rg>  = <CircuitNode@rn, CircuitNode@(rn,rg),float,float>

type node_list<rl,rn>       = < CircuitNode@rn, node_list<rl,rn>@rl >
type wire_list<rl,rw,rn,rg>= < CircuitWire<rn,rg>@rw, wire_list<rl,rw,rn,rg>@rl >

type CircuitPiece<rl,rw,rn> = rr[rpw,rpn,rg]
                            < wire_list<rl,rpw,rpn,rg>@rl, node_list<rl,rpn>@rl >         
                            where rpn <= rn and rg <= rn and rpw <= rw and
                                  rpn * rg and rn * rw and rl * rn and rl * rw

-- Simulation initialization and invocation
function simulate_circuit[rl,rw,rn] ( all_nodes : node_list<rl,rn>@rl, 
                            all_wires : wire_list<rl,rw,rn,rn>@rl, steps : int ), 
      reads(rn,rw,rl), writes(rn,rw,rl) : bool = 
  let pc : <coloring(rn),multicoloring(rn),coloring(rw)> 
            = color_circuit[rn,rw,rl](all_nodes,all_wires) in
  -- Disjoint partition for the owned nodes of each piece
  partition rn using pc.1 as rn0,rn1 in
  -- Aliased partition for ghost nodes of each piece
  partition rn using pc.2 as rg0,rg1 in
  -- Disjoint partition for the owned wires of each piece
  partition rw using pc.3 as rw0,rw1 in
  let lists0 : <wire_list<rl,rw0,rn0,rg0>@rl,node_list<rl,rn0>@rl> = 
        build_lists[rl,rw,rn,rw0,rn0,rg0](all_nodes,all_wires,pc.1,pc.2,pc.3,0) in
  let piece0 : CircuitPiece<rl,rw,rn> = 
        pack lists0 as CircuitPiece<rl,rw,rn>[rw0,rn0,rg0] in
  let lists1 : <wire_list<rl,rw1,rn1,rg1>@rl,node_list<rl,rn1>@rl> =
        build_lists[rl,rw,rn,rw1,rn1,rg1](all_nodes,all_wires,pc.1,pc.2,pc.3,1) in
  let piece1 : CircuitPiece<rl,rw,rn> = 
        pack lists1 as CircuitPiece<rl,rw,rn>[rw1,rn1,rg1] in
      execute_time_steps[rl,rw,rn](piece0,piece1,steps)

-- Time Step Loop
function execute_time_steps[rl,rw,rn] ( p0 : CircuitPiece<rl,rw,rn>, 
      p1 : CircuitPiece<rl,rw,rn>, steps : int ) , reads(rn,rw,rl), writes(rn,rw) : bool = 
  if steps #$<$# 1 then true else
  unpack p0 as piece0 : CircuitPiece<rl,rw,rn>[rw0,rn0,rg0] in 
  unpack p1 as piece1 : CircuitPiece<rl,rw,rn>[rw1,rn1,rg1] in
  let _ : bool = calc_new_currents[rl,rw0,rn0,rg0](piece0.1) in
  let _ : bool = calc_new_currents[rl,rw1,rn1,rg1](piece1.1) in
  let _ : bool = distribute_charge[rl,rw0,rn0,rg0](piece0.1) in
  let _ : bool = distribute_charge[rl,rw1,rn1,rg1](piece1.1) in
  let _ : bool = update_voltage[rl,rn0](piece0.2) in
  let _ : bool = update_voltage[rl,rn1](piece1.2) in
      execute_time_steps[rl,rw,rn](p0,p1,steps-1)

function color_circuit[rn,rw,rl] ( all_nodes : node_list<rl,rn>@rl, 
                               all_wires : wire_list<rl,rw,rn>@rl ), 
        reads(rn,rw,rl) : <coloring(rn), multicoloring(rn), coloring(rw)> =  
  -- Invoke programmer chosen coloring algorithm (e.g. METIS)
  -- return owned, ghost, wire colorings

-- Helper method
function build_lists[rl,rw,rn,rpw,rpn,rg] ( nodes : node_list<rl,rn>@rl, 
       wires : wire_list<rl,rw,rn>@rl, oc : coloring(rn), gc : multicoloring(rn), 
       wc : coloring(rw), c : int), reads(rn,rw,rl), writes(rl) 
       : < wire_list<rl,rpw,rpn,rg>@rl, node_list<rl,rpn>@rl > = 
  -- Construct lists of node and wire pointers for the given colorings
\end{lstlisting}

\begin{lstlisting}[float={t},label={lst:circuit_leaf},caption={Circuilt Leaf Tasks}]
function calc_new_currents[rl,rw,rn,rg] ( ptr_list : wire_list<rl,rw,rn,rg>@rl ), 
      reads(rl,rw,rn,rg), writes(rw) : bool =
  if isnull(ptr_list) then true else
  let wire_node : wire_list<rl,rw,rn,rg> = read(ptr_list) in
  let wire : CircuitWire<rn,rg> = read(wire_node.1) in
  let in_node : CircuitNode = read(wire.1) in
  let out_node : CircuitNode = read(wire.2) in
  let current : float = (in_node.1 - out_node.1) /  wire.3 in 
  let new_wire : CircuitWire<rn,rg> = <wire.1,wire.2,wire.3,current> in
  let _ : CircuitWire<rn,rg>@rw = write(wire_node.1, new_wire) in
      calc_new_currents[rl,rw,rn,rg](wire_node.2)

function distribute_charge[rl,rw,rn,rg] ( ptr_list : wire_list<rl,rw,rn,rg>@rl ), 
      reads(rl,rw,rn), reduces(reduce_charge,rn,rg), atomic(rn,rg) : bool =
  if isnull(ptr_list) then true else
  let wire_node : wire_list<rl,rw,rn,rg> = read(ptr_list) in
  let wire : CircuitWire<rn,rg> = read(wire_node.1) in
  let _ : CircuitNode@rn = reduce(reduce_charge, wire.1, wire.4) in
  let _ : CircuitNode@(rn,rg) = reduce(reduce_charge, wire.2, wire.4) in
      distribute_charge[rl,rw,rn,rg](wire_node.2)

function update_voltage[rl,rn] ( ptr_list : node_list<rl,rn>@rl ), 
      reads(rl,rn), writes(rn) : bool = 
  if isnull(ptr_list) then true else
  let node_node : node_list<rl,rn> = read(ptr_list) in
  let node : CircuitNode = read(node_node.1) in
  let voltage : float = (node.3/node.4) in
  let new_node : CircuitNode = <voltage,node.2,node.3,node.4> in
  let _ : CircuitNode@rn = write(node_node.1, new_node) in
      update_voltage[rl,rn](node_node.2)

-- Reduction function for distribute charge
function reduce_charge ( node : CircuitNode, current : float ) : CircuitNode =
    let new_charge : float = node.3 + current in
        < node.1,new_charge,node.3,node.4>
\end{lstlisting}


