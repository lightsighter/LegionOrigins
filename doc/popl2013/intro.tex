
\section{Introduction}
\label{sec:intro}
Introduction goes here.

% This is a description of how the listings should be formatted.
% It can go anywhere before the listings.
\lstset{
  captionpos=b,
  language=Haskell,
  basicstyle=\scriptsize,
  numbers=left,
  numberstyle=\tiny,
  columns=fullflexible,
  stepnumber=1,
  escapechar=\#,
  keepspaces=true,
  literate={<}{{$\langle$}}1 {>}{{$\rangle$}}1,
  morekeywords={task,rr,int,bool,isnull,partition,as,downregion,upregion,reads,writes,rdwrs,reduces,read,write,reduce,using,unpack,pack,coloring,color,newcolor,atomic,simultaneous},
  deletekeywords={float,head,min,max}
}

% The linked list example
% Add float={t} to the list of options if you want the listing at the top
\begin{lstlisting}[label={lst:linkedlist},caption={Linked List Example}]
type list<r> = < int, list<r>@r >

task sum_list[r] ( l:list<r>@r ), reads(r) : int =
    if isnull(l) then 0 else
    let v : list<r> = read(l) in
        v.1 + sum_list[r](v.2)
\end{lstlisting}

% The tree example with partitioning and tree reversal
\begin{lstlisting}[float={t},label={lst:treereverse},caption={Tree Partitioning and Reversal Example}]
type tree<rl,rls,rt> = rr[rll, rlr, rtl, rtr] 
                       < list<rl>@rls, < tree<rl,rll,rtl>@rtl, tree<rl,rlr,rtr>@rtr > >
     		       where rll #$\le$# rls and rlr #$\le$# rls and rll #$*$# rlr and 
                                  rtl #$\le$# rt and rtr #$\le$# rt  and rtl #$*$# rtr

task color_list[r,rs] ( clr: coloring(rs), head: list<r>@r, 
                        splitval: int ), reads(r) : coloring(rs) =
    if isnull(head) then clr else
    let v : list<r> = read(head) in
    let ps : list<r>@rs = downregion(head, rs) in
        color_list[r,rs](if isnull(ps) then clr 
                                else color(clr, ps, if v.1 #$<$# splitval then 1 else 2), v.2, splitval)

task make_tree[rl,rls,rt] ( l: list<rl>@rl, min: int, 
                            max: int), reads(rl,rt), writes(rt) : tree<rl,rls,rt>@rt =
    if isnull(l) then null tree<rl,rls,rt>@rt else 
    let v : list<rl> = read(l) in
    if (v.1 #$<$# min) | (v.1 #$>$# max) then
    make_tree[rl,rls,rt](v.2, min, max) else
    let root : tree<rl,rls,rt>@rt = new tree<rl,rls,rt>@rt in
    partition rt using newcolor rt as rtl,rtr in
    partition rls using color_list[rl,rls](newcolor rls, v.2, v.1) as rll,rlr in
    let ptr : list<rl>@rls = downregion(l, rls) in
    let lhs : tree<rl,rll,rtl>@rtl = make_tree[rl,rll,rtl](v.2, min, v.1) in
    let rhs : tree<rl,rlr,rtr>@rtr = make_tree[rl,rlr,rtr](v.2, v.1, max) in
    let v2 : tree<rl,rls,rt> = pack < ptr, < lhs, rhs > > as tree<rl,rls,rt>[rll, rlr, rtl, rtr] in
        write(root, v2)

task reverse_tree[rl,rls,rt] ( root: tree<rl,rls,rt>@rt ), reads(rt), writes(rt) : tree<rl,rls,rt>@rt =
     if isnull(root) then null tree<rl,rls,rt>@rt else
     let v : tree<rl,rls,rt> = read(root) in
     unpack v as u : tree<rl,rls,t>[rll, rlr, rtl, rtr] in
     let lhs_r : tree<rl,rll,rtl>@rtl = reverse_tree[rl,rll,rtl](u.2.1) in
     let rhs_r : tree<rl,rlr,rtr>@rtr = reverse_tree[rl,rlr,rtr](u.2.2) in
     let v_r : tree<rl,rls,rt> = pack < u.1, < rhs_r, lhs_r > > as tree<rl,rls,rt>[rlr, rll, rtr, rtl] in
        write(root, v_r)
\end{lstlisting}

\begin{lstlisting}[label={lst:histogram},caption={Histogram with Reductions and Atomic Coherence}]
type hist = < int, int >

task hist_tree1[rl,rls,rt,rh] ( root: tree<rl,rls,rt>@rt, hptr: hist@rh, 
                                t : int ), reads(rt,rls,rh), writes(rh), atomic(rh) : int =
    if isnull(root) then 0 else
    let v : tree<rl,rls,rt> = read(root) in
    unpack v as u : tree<rl,rls,t>[rll, rlr, rtl, rtr] in
    let lval : list<rl> = read(u.1) in
    let hval : hist = read(hptr) in
    let hval2 : hist = if lval.1 #$<$# t then < hval.1 + 1, hval.2 >
                                     else < hval.1, hval.2 + 1 > in
    let _ : hist@rh = write(hptr, hval2) in
        1 + hist_tree1[rl,rll,rtl,rh](u.2.1, hptr, t) + hist_tree1[rl,rlr,rtr,rh](u.2.2, hptr, t)

task inc_bucket ( hval: hist, b: int ) : hist =
    if b #$<$# 2 then < hval.1 + 1, hval.2 > else < hval.1, hval.2 + 1 >

task hist_tree2[rl,rls,rt,rh] ( root: tree<rl,rls,rt>@rt, hptr: hist@rh, 
                                t : int ), reads(rt,rls), reduces(inc_bucket, rh) : int =
    if isnull(root) then 0 else
    let v : tree<rl,rls,rt> = read(root) in
    unpack v as u : tree<rl,rls,t>[rll, rlr, rtl, rtr] in
    let lval : list<rl> = read(u.1) in
    let _ : hist@rh = reduce(inc_bucket, hptr, if lval.1 #$<$# t then 1 else 2) in
        1 + hist_tree1[rl,rll,rtl,rh](u.2.1, hptr, t) + hist_tree1[rl,rlr,rtr,rh](u.2.2, hptr, t)
\end{lstlisting}
